\section{Diskussion}
\label{sec:Diskussion}

Im ersten Aufgabenteil wurde die Charakteristik des Zählrohrs untersucht.
Diese ist mit einem Plateau von etwa $\SI{200}{\volt}$, das etwa $\frac{2}{3}$
vom gesamten Spannungsmessbereich umfasst und einer sehr geringen Plateausteigung
sehr gut.

Die Messungen zu den Nachentladungen und der Totzeit wurden beide jeweils einmal
durchgeführt und sind somit als Stichproben anzusehen.
Die für die Totzeit nach verschiedenen Methoden bestimmten Werte weichen
daher nicht überraschend um $\SI{74.58}{\percent}$ voneinander ab.

Bei der Bestimmung der einzelnen Teilchenladung 
weichen die einzelnen Ladungsmesswerte, wie Abbildung \ref{fig:plot2} zeigt,
recht stark von der Ausgleichsgerade ab. Dies kann an der begrenzt möglich
ablesbaren Stromstärke liegen.


