\section{Diskussion}
\label{sec:Diskussion}

...

Saras Zeug

Denk bitte an /text

...

Für die Position 1 der Puppe weicht das gemessene Trägheitsmoment $I_\text{1,mess}$ um 
$\SI{340}{\percent}$ vom theoretischen Trägheitsmoment $I_\text{1,theo}$ ab. Für Position 2
weicht $I_\text{2,mess}$ um $\SI{42}{\percent}$ von $I_\text{2,theo}$ ab.

Dies ist einerseits durch die geringe Periodendauer der schwingenden Puppe zu erklären,
da es schwieriger ist, diese zu messen, als die Periodendauern der ausgedehnteren Körper.

Andererseits fließen in die theoretisch berechneten Trägheitsmomente sehr viele Messfehler
der Ausmessungen und somit der Volumina der einzelnen Komponenten, sowie deren Abstände 
von der Drehachse ein.
Letzteres wird auch durch die Schätzung der einzelnen Schwerpunkte erschwert. 
Auch ist die Näherung der einzelnen Komponenten durch geometrische Figuren nicht
perfekt möglich.
