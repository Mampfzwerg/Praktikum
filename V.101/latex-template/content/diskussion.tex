\section{Diskussion}
\label{sec:Diskussion}

Im erstem Versuchsteil wurden die spezifischen Kenngrößen der verwendeten 
Apparatur ermittelet. Der ermittelte Wert der Winkelrichtgröße $D$ weist dabei 
lediglich eine statistische Abweichung von $\SI{6.13}{\percent}$ auf und das 
Eigenträgheitsmoment der Drillachse $I_\text{D}$ ist mit einem statistischen 
Fehler von $\SI{44.44}{\percent}$ behaftet. Letzterer Fehler ist relativ groß 
und damit zu erklären, dass die als masselos angenommene Stange zu 
schwer ist, als das diese vernachlässigt werden könnte. Die Stange hatte eine 
Länge von $\SI{0.6}{\meter}$ und ein Gewicht von $\SI{96.6}{\gram}$. Damit ergibt
sich ein Trägheitsmoment von $\SI{2.90e-3}{\kilo\gram\meter²}$, was in etwa so groß
ist wie das ermittelte Trägheitsmoment der Drillachse. Dadurch wird klar, dass die
Stange nicht vernachlässigt hätte werden dürfen. \\
Die Verifikation des Steiner’schen Satzes erfolgte ebenfalls in diesem Versuchsteil. Der
theoretische, lineare Zusammenhang wird gut durch die Messdaten bestätigt. Allerdings weicht
die Steigung der Ausgleichsgeraden um $\SI{20.69}{\percent}$ von dem theoretisch zu erwartenden 
Wert ab. \\
Die im zweiten Versuchsteil ermittelten Trägheitsmomente von zwei Körpern weichen nur um
$\SI{6.9}{\percent}$ bei der Kugel und um $\SI{0.625}{\percent}$ bei dem Zylinder von den 
Theoriewerten ab. Demnach eignet sich die Apparatur gut zur Bestimmung der Trägheitsmomente
von Körpern. 
Für die Position 1 der Puppe weicht das gemessene Trägheitsmoment $I_\text{1,mess}$ um 
$\SI{340}{\percent}$ vom theoretischen Trägheitsmoment $I_\text{1,theo}$ ab. Für Position 2
weicht $I_\text{2,mess}$ um $\SI{42}{\percent}$ von $I_\text{2,theo}$ ab.
Dies ist einerseits durch die geringe Periodendauer der schwingenden Puppe zu erklären,
da es schwieriger ist, diese zu messen, als die Periodendauern der ausgedehnteren Körper.
Andererseits fließen in die theoretisch berechneten Trägheitsmomente sehr viele Messfehler
der Ausmessungen und somit der Volumina der einzelnen Komponenten, sowie deren Abstände 
von der Drehachse ein.
Letzteres wird auch durch die Schätzung der einzelnen Schwerpunkte erschwert. 
Auch ist die Näherung der einzelnen Komponenten durch geometrische Figuren nicht
perfekt möglich.
