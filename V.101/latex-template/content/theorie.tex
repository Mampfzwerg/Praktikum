\section{Theorie}
\label{sec:Theorie}

Ziel des Versuches ist es, die Trägheitsmomente verschiedener Körper 
zu messen und den Steiner'schen Satz zu verifizieren.

Durch das Drehmoment $M$, das Trägheitsmoment $I$ und die Winkelbeschleunigung 
$\dot\omega$ wird die Dynmaik von Rotatoren beschrieben. Das Trägheitsmoment
einer Punktmasse ist mit $I = m\cdot r²$ gegeben. Bei mehreren Massenpunkten
geht dies in eine Summe oder bei einer kontinuierlichen Masseverteilung 
in ein Intergral über. Sowohl das Trägheitsmoment als auch das Drehmoment 
sind dabei im Allgemeinem von der Lage des Körpers zur Drehachse abhängig. 
Wenn die Drehachse nicht durch den Schwerpunkt verläuft, so lässt sich das
Trägheitsmoment mit dem Steiner'schen Satz berechnen: 

\begin{equation}
I = I_S + ma² 
\label{Steiner}
\end{equation}

Dabei ist $I_S$ das Trägheitsmoment des Körpers, wenn die Drehachse mit der 
Schwerpunktsachse zusammenfällt und $a$ der Abstand der Dreh- zur Schwerpunktsachse, 
wenn dies nicht der Fall ist. Außerdem ist $m$ die Gesamtmasse des Körpers.
Greift an einem Körper die Kraft $\vec{F}$ am Ort $\vec{r}$ an, so wirkt 
an diesem Körper das Drehmoment $\vec{M} = \vec{F} \times \vec{r}$. Sollte es
sich um ein Schwingungsfähiges System handeln, so führt eine Auslenkung des 
Körpers aus seiner Ruhelage zu einem rücktreibendem Drehmoment. Für kleine
Winkel schwingt ein solche harmonischer Oszillator näherungsweise mit 
der Periodendauer

\begin{equation}
T = 2\pi\sqrt{\frac{I}{D}}.
\label{eqn:Periode}
\end{equation}

Dabei ist $D$ die Winkelrichtgröße und $I$ das Trägheitsmoment. Die 
Winkelrichtgröße $D$ hängt mit dem Drehmoment über die Beziehung 

\begin{equation}
M = D\cdot\varphi
\label{eqn:Winkelrichtgröße}
\end{equation}

zusammen. Die Winkelrichtgröße gibt dabei an, welches Drehmoment bei einer 
Auslenkung um den Winkel $\varphi$ wirkt und kann mit zwei Methoden bestimmt 
werden. Zum einen die statistische und zum anderen die dynamische Methode. 
Bei der statistischen Methode wird der Auslenkwinkel $\varphi$ als Funktion
der wirkenden Kraft bzw. des Drehmoments bestimmt. Bei der dynamischen 
Methode wird $D$ über die Periodendauer mit der Beziehung \eqref{eqn:Periode}
bestimmt. Hierbei ist zu beachten, dass $I$ bekannt sein muss. 
Durch Kombination dieser beiden Methoden können $I$ und $D$ gleichzeitig
bestimmt werden. 

%\cite{sample}
