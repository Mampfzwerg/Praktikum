\section{Auswertung}
\label{sec:Auswertung}

\subsection{Bestimmung der Winkelrichtgröße}

Die aufgenommen Messdaten zur Bestimmung der Winkelrichtgröße sind in Tabelle \ref{tab:Messdaten1}
zu finden. Dabei wurde die Kraft bei verschiedenen Auslenkwinkeln $\varphi$ gemessen. Der Hebelarm
hat die konstante Länge \SI{0.3}{\meter}.

\begin{table}
\centering
\caption{Ausgelenkter Winkel mit dazugehöriger Kraft}
\label{tab:Messdaten1}
\sisetup{table-format=2.1}
\begin{tabular}{c c c c}
\toprule
$Winkel \,/\, °$ & $Winkel \,/\, rad$ & $F \,/\, \si{\milli\newton}$ & $M \,/\, \si{\milli\newton\meter}$\\
\midrule
 35 & 0.61 &  30 &  9.00\\
 45 & 0.79 &  35 & 10.05\\
 50 & 0.87 &  40 & 12.00\\
 55 & 0.96 &  50 & 15.00\\
 60 & 1.05 &  55 & 16.50\\
 70 & 1.22 &  61 & 18.30\\
 80 & 1.40 &  83 & 24.90\\
 90 & 1.57 & 102 & 30.60\\
100 & 1.66 & 110 & 32.40\\
 95 & 1.74 & 108 & 33.00\\
\bottomrule
\end{tabular}
\end{table}

Anhand von Formel \eqref{eqn:Winkelrichtgröße} ist zu erkennen, dass die Winkelrichtgröße $D$ 
lediglich ein Proportionalitätsfaktor zwischen Drehmoment $M$ und Winkel $\varpi$ ist. Zur Ermittlung 
dieser Größe wird eine lineare Regressions mittels python und matplotlib durchgeführt. 
Das Ergebnis ist in Abbildung \ref{fig:plot1} zu sehen. 


\begin{figure}
  \centering
  \includegraphics[scale=0.8]{content/plot1.pdf}
  \caption{Darstellung des Zusammenhanges zwischen M und $\varphi$}
  \label{fig:plot1}
\end{figure}

Die Ausgleichsgerade ist gegeben mit $M(\varphi) = a\cdot \varphi + b$. Für die Parameter Regressionsparameter
ergibt sich: 

\begin{align*}
a &= \SI{23.34+-1.43}{\milli\newton\meter} \\
b &= \SI{-7.5+-1.78}{\milli\newton\meter}
\end{align*}

Der Steigungsfaktor $b$ entspricht dabei der Winkelrichtgröße $D$. Somit ergibt sich: 

\begin{equation*}
D = \SI{23.34+-1.43}{\milli\newton\meter}
\end{equation*}

\subsection{Bestimmung des Trägheitsmoments der Drillachse}

Zur Bestimmung des Eigenträgheitsmoments der Drillachse wird die Periodendauer $T$ in Abhängigkeit 
des Abstands $a$ der Gewichte zur Drillachse gemessen. Die erhaltenen Daten sind in Tabelle 
\ref{tab:Messdaten2} aufgeführt. Dabei ist $T_7$ die gemessene Dauer für 7 Perioden.

\begin{table}
\centering
\caption{Messwerte zum Trägheitsmoment der Drehachse}
\label{tab:Messdaten2}
\sisetup{table-format=2.1}
\begin{tabular}{c c c c c}
\toprule
$a \,/\, \si{\meter}$ & $a² \,/\, m²$ & $T_7\,/\, \si{\second}$ & $T \,/\, \si{\second}$ & $T² \,/\, s²$\\
\midrule
 0.2700 & 0.0729 & 57.93\,\pm 0.5 & 8.28\,\pm 0.071 & 68.49\,\pm 1.18\\
 0.2300 & 0.0529 & 50.30\,\pm 0.5 & 7.19\,\pm 0.071 & 51.63\,\pm 1.03\\
 0.2090 & 0.0437 & 46.89\,\pm 0.5 & 6.70\,\pm 0.071 & 44.87\,\pm 0.96\\
 0.1885 & 0.0355 & 43.00\,\pm 0.5 & 6.14\,\pm 0.071 & 37.73\,\pm 0.88\\
 0.1695 & 0.0287 & 40.10\,\pm 0.5 & 5.73\,\pm 0.071 & 32.82\,\pm 0.82\\
 0.1490 & 0.0222 & 35.72\,\pm 0.5 & 5.10\,\pm 0.071 & 26.04\,\pm 0.73\\
 0.1295 & 0.1677 & 32.58\,\pm 0.5 & 4.65\,\pm 0.071 & 21.66\,\pm 0.66\\
 0.1090 & 0.0119 & 28.85\,\pm 0.5 & 4.12\,\pm 0.071 & 16.99\,\pm 0.59\\
 0.0890 & 0.0079 & 25.49\,\pm 0.5 & 3.64\,\pm 0.071 & 13.26\,\pm 0.52\\
 0.0690 & 0.0048 & 22.36\,\pm 0.5 & 1.19\,\pm 0.071 & 10.20\,\pm 0.46\\
\bottomrule
\end{tabular}
\end{table}

Aus \eqref{eqn:Steiner} und \eqref{eqn:Periode} folgt der Zusammenhang: 

\begin{equation*}
T² = \underbrace{\frac{4\pi²m}{D}}_{\hat = s}\cdot a² + \underbrace{\frac{4\pi²}{D} I_D}_{\hat = b}
\end{equation*}

Mittels linearer Regression wird nun der als $b$ bezeichnete Parameter 
ermittelt und daraus das gesuchte Trägheitsmoment $I_D$ berechnet. Das 
Ergebnis ist in Abbildung \ref{fig:plot2} dargestellt. 

\begin{figure}
  \centering
  \includegraphics[scale=0.8]{content/plot2.pdf}
  \caption{$T²$ gegen $a²$ aufgetragen mit eingezeichnetere Regression}
  \label{fig:plot2}
\end{figure}

Die Parameter der Ausgleichsgerade ergeben sich zu: 

\begin{align*}
s &= \SI{905.17+-54.89}{\second²\per\meter²}\\
b &= \SI{4.59+-1.99}{\second²}
\end{align*}

Damit berechnet sich $I_D$ zu: 

\begin{equation*}
I_D = \frac{b\cdot D}{4\pi²} = \SI{2.7+-1.2e-3}{\kilo\gram\meter²}
\end{equation*}

Der statistische Fehler berechnet sich dabei mit der Gaußschenfehlerfortpflanzung 
zu: 

\begin{equation*}
\Delta I_D = \sqrt{\left(\frac{\partial I_D}{\partial b}\cdot \Delta b\right)²+\left(\frac{\partial I_D}{\partial D}\cdot \Delta D\right)²}
= \sqrt{\left(\frac{D}{4\pi²}\cdot \Delta b\right)²+\left(\frac{b}{4\pi²}\cdot \Delta D\right)²}
\end{equation*}

\subsection{Trägheitsmomente zweier einfacher Körper}

Die zu untersuchenden Körper werden oben auf der Drillachse befestigt, zum Schwingen gebracht 
und die Periodendauer $T$ gemessen. Es werden zwei Körper untersucht: Ein Zylinder und eine Kugel.
Diese Körper besitzen folgende Maße: 

\begin{align*}
m_Z &= \SI{1005.9}{\gram}\\
l_Z &= \SI{139.9+-0.05}{\centi\meter}\\
d_Z &= \SI{79.95+-0.05}{\centi\meter}\\
m_K &= \SI{812.4}{\gram}\\
r_K &= \SI{7.5}{\centi\meter} 
\end{align*}

Die ermittelten Periodendauer und Trägheitsmomente finden sich in Tabelle ....

\begin{table}
\centering
\caption{Messwerte Kugel und Zylinder}
\label{tab:Messdaten2}
\sisetup{table-format=2.1}
\begin{tabular}{c c c c}
\toprule
\multicolumn{2}{c}{Zylinder} &  \multicolumn{2}{c}{Kugel} \\
\midrule
$T_Z \,/\, \si{\second}$ & $I_Z \,/\, \SI{1e-3}{\kilo\gram\meter²}$ & $T_K\,/\, \si{\second}$ & $I_K \,/\, \SI{1e-3}{\kilo\gram\meter²}$\\
\midrule
 1.21\,\pm 0.07 & 0.862\,\pm 0.115 & 1.73\,\pm 0.71 & 1.769\,\pm 0.182\\
 1.14\,\pm 0.07 & 0.770\,\pm 0.107 & 1.69\,\pm 0.71 & 1.691\,\pm 0.176\\
 1.15\,\pm 0.07 & 0.782\,\pm 0.108 & 1.73\,\pm 0.71 & 1.767\,\pm 0.182\\
 1.15\,\pm 0.07 & 0.782\,\pm 0.108 & 1.66\,\pm 0.71 & 1.629\,\pm 0.172\\
 1.16\,\pm 0.07 & 0.196\,\pm 0.109 & 1.69\,\pm 0.71 & 1.686\,\pm 0.176\\
\midrule
\multicolumn{2}{c}{$\bar{I_Z}= \SI{0.80+-0.07}{\kilo\gram\meter²}$} & \multicolumn{2}{c}{$\bar{I_K} = \SI{1.71+-0.12}{\kilo\gram\meter²}$} \\
\bottomrule
\end{tabular}
\end{table}