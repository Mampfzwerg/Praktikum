\section{Diskussion}
\label{sec:Diskussion}

Der aus Plot \ref{fig:plot2} bestimmte Quotient aus der Planckschen Konstante und der Elemtarladung lautet 

\begin{equation*}
\left(\frac{h}{e_0}\right)_\text{gem} = \SI{1.59+-1.23e-15}{\volt\second}.
\end{equation*}

Verglichen mit dem Literaturwert [1],

\begin{equation*}
\left(\frac{h}{e_0}\right)_\text{lit} = \SI{4.136e-15}{\volt\second},
\end{equation*}

ergibt sich eine prozentuale Abweichung von

\begin{equation*}
\Delta\left(\frac{h}{e_0}\right) = \SI{61.56}{\percent}.
\end{equation*}

Diese Abweichung ist sehr groß. Dies liegt vermutlich daran, dass nur 4 Werte 
verwendet wurden um $h$ mit einer linearen Ausgleichsrechnung zu bestimmen, worunter
natürlich die Genauigkeit leidet. 
Die aus jenem Plot berechnete Austrittsarbeit beträgt 

\begin{equation*}
E_\text{A} = \SI{0.264+-0.750}{\eV}.
\end{equation*}

Nach [2] kann vermutet werden, dass es sich bei dem verwendeten Metall am ehesten um 
Caesium handelt, allerdings ist auch dieser Wert noch recht weit von unserer berechneten
Austrittsarbeit entfernt. Die Abweichung kann durch die gleiche Argumentation erklärt werden
wie der zur Abweichung von $\left(\frac{h}{e_0}\right)_\text{gem}$ zu $\left(\frac{h}{e_0}\right)_\text{lit}$.
Es bleibt zu erwähnen, dass während der Durchführung des Experiments auffiel, dass das Picoamperemeter
starken Schwankungen unterlad und somit die Bestimmung eines Wertes nicht exakt möglich war. 