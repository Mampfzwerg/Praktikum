\section{Auswertung}
\label{sec:Auswertung}


\subsection {Bestimmung der Wärmekapazität $c_g m_g$}

Zunächst wird die Wärmekapazität $c_g m_g$ gemäß Kapitel \ref{sec:Durchführung}
bestimmt. Die Messwerte sind in Tabelle \ref{tab:Kalorimeter} aufgeführt.

\begin{table}
\centering
\caption{Messwerte zur Bestimmung von $c_g m_g$}
\label{tab:Kalorimeter}
\sisetup{table-format=2.1}
\begin{tabular}{c c c}
\toprule
$ $ & $T \,/\, \si{\kelvin}$& $m \,/\, \si{\gram}$\\
\midrule
X: & 295.25 & 157.64\\
Y: & 349.15 & 155.22\\
M: & 317.95 & 312.86\\
\bottomrule
\end{tabular}
\end{table}

Nun lässt sich mit \ref{eqn:KapKalori} die Wärmekapazität $c_g m_g$ bestimmen
zu: 

\begin{equation*}
c_g m_g = \SI{232.83}{\joule\per\kelvin}
\end{equation*}

Dabei ist die spezifische Wärmekapazität des Wassers bei 40°C gegeben durch

\begin{equation}
c_w = \SI{4.18}{\joule\per\gram\per\kelvin} [1]
\end{equation}

\subsection{Molwärme verschiedener Stoffe}

\subsubsection{Aluminium}

Die Messwerte für Aluminium sind in Tabelle \ref{tab:Aluminium} aufgeführt.

\begin{table}
\centering
\caption{Messwerte von Aluminium}
\label{tab:Aluminium}
\sisetup{table-format=2.1}
\begin{tabular}{c c c}
\toprule
$ $ & $T \,/\, \si{\kelvin}$& $m \,/\, \si{\gram}$\\
\midrule
  Wasser: & 296.95 & 509.88\\
  Körper: & 373.15 & 113.86\\
Mischung: & 300.55 & 623.74\\
\bottomrule
\end{tabular}
\end{table}

Nun wird zunächst die Wärmekapazität $c_k$ gemäß \ref{eqn:KapProbe} berechnet. 
Es ergibt sich:

\begin{equation*}
c_k = \SI{1.03}{\joule\per\gram\per\kelvin}
\end{equation*}

Für die Molwärme $C_V$ ergibt sich nach \ref{eqn:Molwaerme2}.

\begin{equation*}
C_V = \SI{27.7}{\joule\per\kelvin\per\mol}
\end{equation*}

Die körperspezifischen Konstanten sind dabei gegeben durch: 

\begin{align*}
M &= \SI{27.0}{\gram\per\mol}\\
\alpha &= \SI{23.5e-6}{\per\kelvin}\\
\kappa &= \SI{75e9}{\newton\per\m²}\\
\rho &= \SI{2.7}{\gram\per\centi\meter³} [1]
\end{align*}

Dies ergibt eine Abweichung von \SI{0.87}{\percent} von den entsprechenden $3R$ aus 
dem Dulong-Petitschen Gesetz. 

\subsubsection{Kupfer}

Die Messwerte für Kupfer sind in Tabelle \ref{tab:Kupfer} aufgeführt.

\begin{table}
\centering
\caption{Messwerte von Kupfer}
\label{tab:Kupfer}
\sisetup{table-format=2.1}
\begin{tabular}{c c c c}
\toprule
$ $ & $ $ & $T \,/\, \si{\kelvin}$& $m \,/\, \si{\gram}$\\
\midrule
1.)  &  Wasser: & 296.65 & 514.09\\
     &  Körper: & 372.85 & 237.07\\
     &Mischung: & 300.95 & 751.16\\
     \midrule
2.)  &  Wasser: & 296.95 & 514.21\\
     &  Körper: & 373.15 & 237.07\\
     &Mischung: & 300.55 & 751.28\\
     \midrule
3.)  &  Wasser: & 296.55 & 514.11\\
     &  Körper: & 373.15 & 237.07\\
     &Mischung: & 300.35 & 751.18\\

\bottomrule
\end{tabular}
\end{table}

Aus diesen Werten lässt sich zunächst die Wärmekapazität $c_k$ berechnen. 
Aus $c_k$ wird der Mittelwert und dessen Fehler bestimmt. Diese und die 
Mischtemperaturen samt Mittelwert und dessen Fehler sind in Tabelle 
\ref{tab:Ergebnisse} aufgeführt.

\begin{table}
\centering
\caption{Ergebnisse Kupfer}
\label{tab:Ergebnisse}
\sisetup{table-format=2.1}
\begin{tabular}{c c c}
\toprule
$ $ & $c_k \,/\, \si{\joule\per\gram\per\kelvin}$& $T \,/\, \si{\kelvin}$\\
\midrule
    1.)  & 0.601 & 300.95\\
    2.)  & 0.498 & 300.55\\
    3.)  & 0.524 & 300.35\\
\midrule
Mittel:  & 0.541\,\pm\,0.044 & 300.62\,\pm\,0.25\\
\bottomrule
\end{tabular}
\end{table}

Der Mittelwert ergibt sich nach 

\begin{equation*}
\overline x = \frac{1}{n} \sum^{n}_{i=1} x_i
\end{equation*}

und dessen Fehler nach 

\begin{equation*}
\Delta \overline x = \sqrt{\frac{\sum^{n}_{i=1}(x_i - \overline x)^2}{n(n-1)}}. 
\end{equation*}

Nun wird aus $\overline c_k$ und $\overline T$ die Molwärme $C_V$ nach 
\ref{eqn:Molwaerme2} berechnet. Für die Berechnung werden die 
körperspezifischen Konstanten benötigt. $V_o$ lässt sich aus der Dichte 
$\rho$ und molaren Masse $M$ zu $V_o = \frac{M}{\rho}$ berechnen.

\begin{align*}
M &= \SI{63.5}{\gram\per\mol}\\
\alpha &= \SI{16.8e-6}{\per\kelvin}\\
\kappa &= \SI{136e9}{\newton\per\m²}\\
\rho &= \SI{8.96}{\gram\per\centi\meter³}\\
V_o &= \SI{709e-6}{\cubic\meter\per\mol} [1]
\end{align*}

Der Fehler von $C_V$ berechnet sich gemäß der Gaußschen Fehlerfortpflanzung
zu:

\begin{align*}
\Delta C_V &= \sqrt{\frac{\partial C_V}{\partial c_k}\cdot\Delta c_k + \frac{\partial C_V}{\partial T}\cdot\Delta T}\\
&= \sqrt{M\cdot\Delta c_k - 9\cdot\alpha²\kappa V_o}
\end{align*}

Schließlich ergibt sich für die Molwärme: 

\begin{equation*}
C_V = \SI{39.3+-2.8}{\joule\per\kelvin\per\mol}
\end{equation*}

Dies ergibt eine Abweichung von \SI{28.91}{\percent} von den entsprechenden $3R$ aus 
dem Dulong-Petitschen Gesetz. 

