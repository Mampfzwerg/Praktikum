\section{Theorie}
\label{sec:Theorie}

Bei diesem Versuch soll die Molwärme verschiedener Festkörper
ermittelt werden und daraus geschlossen werden, ob das 
Dulong-Petitsche Gesetz als gute Nährung angesehen werden kann
oder eine quantenmechanische Betrachtung sinnvoller ist. 

\subsection{Die Molwärme}
\label{sec:Molwaerme}

Unter dem Begriff Molwärme $C$ versteht man die Wärmemenge $\diff Q$, 
die von einem Stoff aufgenommen werden muss, um ein Mol 
dieses Stoffes um $\diff T$ zu erhitzen. Daraus ergibt sich der 
Zusammenhang:

\begin{equation*}
C = \frac{\diff Q}{\diff T}
\end{equation*}

Wenn keine Arbeit an dem Stoff verrichtet wird, ergibt sich
nach dem ersten Haupsatz der Thermodynamik die Gleichheit 
von Wärmemenge und Energie. 

\begin{equation*}
\diff U = \diff Q + \diff A
\end{equation*}

Die Wärmemenge kann unter verschiedene Bedingungen zugeführt 
werden. In diesem Zusammenhang wird zwischen konstanten 
Druckverhältnissen und konstantem Volumen des Testkörpers 
unterschieden. Letztere Bedingung führt zu: 

\begin{equation}
C_V = \left(\frac{\diff Q}{\diff T}\right)_V = \left(\frac{\diff U}{\diff T}\right)_V
\label{eqn:Molwaerme}
\end{equation}

Das Dulong-Petitsche Gesetz sagt nun aus, dass ein Stoff 
unabhängig vom chemischen Charakter stets eine Molwärme von
$C_V = 3 R$ hat. Dabei ist R die allgemeine Gaskonstante mit 
$R = \SI{9.3144621}{\joule\per\mol\per\kelvin}$ [1].

\subsection{Klassische Berechnung der Molwärme}
\label{sec:Klassisch}

Für die klassische Berechnung der Molwärme wird ein Festkörper
als eine regelmäßige Anordnung von Atomen modelliert, welche 
harmonisch schwingen. Diese harmonischen Oszillatoren haben je 
drei Freiheitsgrade (je einer in x-,y- oder z-Richtung). Somit 
tragen die Oszillatoren zur Gesamtenergie bzw. zu der im Körper
gespeicherten Wärme bei.
Das Äquipartitionstheorem der klassischen Thermodynamik besagt,
dass jedes Molekül pro Freiheitsgrad im thermischen 
Gleichgewicht eine mittlere kinetische Energie von 
$E_{kin} = \frac{1}{2} k_B T$ trägt. Somit ist die mittlere
kinetische Energie jedes Atoms gegeben durch

\begin{equation}
E_ {kin} = \frac{3}{2} k_B T.
\end{equation}

Außerdem trägt auch die potentielle Energie der einzelnen 
Freiheitsgrade zur Gesamtenergie des Festkörpers bei. 
Im zeitlichen Mittel hat diese genau den gleichen Betrag, 
wie die jeweilige kinetische Energie. Dies folgt aus dem 
Virialsatz. Dieser besagt, dass das Potential des 
eindimensionalen harmonischen Oszillators proportional zum
Quadrat der Auslenkung ist. Somit ergibt sich für die
gesamte Energie eines Atomes:

\begin{equation*}
E_{ges} = E_ {kin} + E_ {pot} = 2\cdot E_ {kin} = 3 k_B T 
\end{equation*}

Multipliziert man dieses Ergebnis nun mit der Avogadrokonstante
$N_A$, so erhält man die gesamte Energie in einem Mol eines 
Stoffes: 

\begin{equation}
U = 3\underbrace{N_A k_B}_{=R} T = 3 R T 
\label{eqn:innereEnergie}
\end{equation}

Nach \ref{eqn:Molwaerme} ergibt sich somit für die Molwärme:

\begin{equation*}
C_V = 3 R 
\end{equation*}

\subsection{Quantenmechanische Ergänzungen}
\label{sec:Quanten}

Auffällig ist, dass bei tiefen Temperaturen die klassische 
Betrachtung nicht mehr hinreichend zutreffend ist. Dies 
lässt sich mit der Quantenmechanik erklären.
Die entscheidende Änderung liegt darin, dass die mit der 
Frequenz $\omega$ schwingenden Atome gemäß der Quantentheorie
ihre Gesamtenergie nur um diskrete Werte

\begin{equation*}
\Delta E = n \hbar \omega
\end{equation*}

ändern können. Daraus ergibt sich, dass die Energie nun
nicht mehr proportional zur Temperatur ist.
Zur Berechnung der Energie muss über alle möglichen 
Energiezustände, gewichtet mit der Wahrscheinlichkeit
ihres Auftretens, summiert werden. Die Wahrscheinlichkeit
ist durch die Boltzmann-Verteilung festgelegt: 

\begin{equation*}
W(E) = \frac{e^{\left(-\frac{E}{k_B T}\right)}}{k_B T}
\end{equation*}

Damit ergibt sich:

\begin{equation*}
U = \frac{3 N_A \hbar \omega}{e^{\frac{\hbar \omega}{k_B T} - 1}}
\end{equation*}

Zur Untersuchung dieses Terms wird eine Taylorentwicklung
der Exponentialfunktion durchgeführt. 

\begin{equation*}
U = \frac{3 N_A \hbar \omega}{1 + \frac{\hbar\omega}{k_B T} + \mathcal{O}(T^{-2})-1}
= \frac{3 N_A k_B T}{1 + \mathcal{O} (T^{-2})}
= \frac{3 R T}{1 + \mathcal{O} (T^{-2})}
\end{equation*}

So erkennt man sehr einfach, dass die somit erhaltene innere
Energie kleiner ist, als die in (\ref{eqn:innereEnergie}) 
betrachtete. Somit ist auch die Molwärme geringer als der
klassische Wert von $3 R$. Wenn jedoch im Fall $T \rightarrow \infty$
der Restterm verschwindet, ergibt sich wieder der klassische
Wert.
Man erkennt außerdem, dass dieser Grenzfall für schwere Atome
bei niedrigeren Temperaturen bereits erreicht wird. Dies 
liegt daran, dass der Exponent der e-Funktion wegen 
$\omega \propto \frac{1}{\sqrt{m}}$ für große Massen kleiner
wird und somit die Taylorentwicklung eine bessere Nährung 
darstellt.

\subsection{Mischungskalometrie}

Da die Konstanz des Volumens nur unter enormen Drücken 
gewehrleistet werden kann, wird stattdessen der Druck 
konstant gehalten. 
Die Messung der spezifischen Wärmekapazität $c_k$ eines 
Festkörpers bei p = const kann mit einem Mischungskalorimeter
ermittelt werden.
Dazu wird ein aufgeheizter Probekörper der Masse $m_k$ und 
Temperatur $T_k$ in ein mit Wasser der Masse $m_w$ gefülltes 
Dewargefäß getaucht. Zuvor sind Wasser und Gefäß in einem
thermischen Gleichgewicht mit der Temperatur $T_w$. 
Nach dem Eintauchen stellt sich erneut ein thermisches 
Gleichgewicht mit der Mischtemperatur $T_m$ ein. 
Setzt man voraus, dass es sich um ein abgeschlossenes 
System handelt und an diesem keine mechanische Arbeit 
verrichtet wird, oder Energie entweicht, kann mit der 
Formel 

\begin{equation}
c_k = \frac{(c_w m_w + c_g m_g)(T_m -T_w)}{m_k(T_k - T_m)}
\label{eqn:KapProbe}
\end{equation}

die spezifische Wärmekapazität $c_k$ bei konstantem Druck
berechnet werden. Dabei bezeichnet $c_w$ die spezifische
Wärmekapazität des Wassers und $c_g m_g$ die Wärmekapazität
des Kalorimeters.
Dabei können alle Werte direkt gemessen werden, außer
$c_g m_g$. Um diesen Wert zu ermitteln, müssem im Kalorimeter
zwei Wassermengen mit Massen $m_x$ und $m_y$ und Temperaturen
$T_x$ und $T_y$ zu der Mischtemperatur $T_m\prime$ vermischt werden.
Dann gilt: 

\begin{equation}
c_g m_g = \frac{c_w m_y(T_y - T_m\prime)- c_w m_x(T_m\prime - T_x)}{T_m\prime - T_x}
\label{eqn:KapKalori}
\end{equation}

Wenn $c_k$ bekannt ist, kann daraus die Molwärme $C_V$ bei 
konstantem Volumen gemäß 

\begin{equation}
C_V = \underbrace{c_k \cdot M}_{=c_p} - 9 \alpha ^2 \kappa V_o T
\label{eqn:Molwaerme2}
\end{equation}

berechnet werden. Hierbei sind $\alpha$, $\kappa$ und $V_o$
körperspezifische Konstanten. Die Größe $\alpha$ wird als
linearer Ausdehnungskoeffizient, $\kappa$ als Kompressionsmodul
und $V_o$ als molares Volumen bezeichnet.

\cite{sample}
