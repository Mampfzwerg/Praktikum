\section{Durchführung}
\label{sec:Durchführung}

Um die spezifischen Wärmekapazitäten $C_V$ der beiden Festkörperproben (Aluminium und Kupfer)
genau bestimmen zu können,
muss in einer vorbereitenden Messung erst die Wärmekapazität des Kalorimeters $c_gm_g$ bestimmt werden,
um die an die Wände des Kalorimeters abgegebene Wärmemenge in die Ermittlung der Wärmekapazitäten der
Festkörperproben einbeziehen zu können.

Dazu müssen, wie aus Gleichung \ref{eqn:KapKalori} ersichtlich,
erst die Massen $m_x$, $m_y$ der beiden Wasserproben mittels einer feinen Waage bestimmt werden.
Dabei ist darauf zu achten, dass die Masse der Bechergläser nicht mit eingerechnet wird. 
Anschließend wird eine der Wasserproben auf einer Heizplatte erhitzt und schließlich werden die
Temperaturen $T_x$, $T_y$ der Proben mithilfe eines Thermoelements gemessen.
Zuletzt werden beide Wasserproben in das Kalorimeter gegossen und die sich einstellende Mischtemperatur 
$T_m$ gemessen, sobald sich das thermische Gleichgewicht eingestellt hat.
Zur Beschleunigung des Vorgangs wird ein Magnetrührer verwendet.

Die verbleibende Größe, die Wärmekapazität des Wassers $c_w$, muss nicht bestimmt werden, sondern ist mit
$\SI{4.18}{\joule\per\gram\cdot\kelvin}$ bei ca. $\SI{40}{\celsius}$ Wassertemperatur angegeben.


Nun wird eine, ähnlich zu der zuvor beschriebenen, Messung zur Bestimmung der spezifischen Wärmekapazitäten
$C_V$ von Aluminium und Kupfer durchgeführt. Hierzu muss zunächst jeweils die Masse $m_k$ des Probekörpers
und die Masse $m_w$ des Wassers mit einer feinen Waage bestimmt werden.
Anschließend wird die Probe in einem Wasserbad auf einer Heizplatte auf ca. $\SI{100}{\celsius}$ erhitzt. 
Daraufhin wird die Probe dem Wasserbad entnommen und die Temperaturen $T_k$ und $T_w$ des Probekörpers
und des Wassers gemessen.
Schließlich werden das Wasser und die Probe in das Kalorimeter gegeben und erneut wird die sich einstellende
Mischtemperatur $T_m$ gemessen.

Die zweite Messung wird für Aluminium einmal, für Kupfer dreimal durchgeführt. Aus den gemessenen
Größen kann nun nach Gleichung \ref{eqn:KapProbe} die jeweilige spezifische Wärmekapazität des Stoffes
berechnet werden.









