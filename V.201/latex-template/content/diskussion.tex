\section{Diskussion}
\label{sec:Diskussion}

Die Abweichungen der gemessenen Molwärmen von den durch das Dulong-Petitsche 
Gesetz erwarteten $3R$ lassen keine Aussage über die Richtigkeit der 
klassischen Betrachtung zu. Dies ist so, da die Messungen mit großen
Unsicherheiten behaftet sind. Eine mögliche Fehlerquelle ist bereits
die Annahme bei Formel (\ref{eqn:KapKalori}) und (\ref{eqn:KapProbe}), 
dass die Wärmemengen eins zu eins ineinander übergehen. Diese Annahme 
würde gelten, wenn das System vollkommen abgeschlossen wäre. Dies kann
allerdings nicht komplett gewährleistet werden, da vermutlich auch Wärme an 
die Umgebungsluft abgegeben worden ist. 
Auch wurde während der gesamten Messung die spezifische Wärmekapazität 
von Wasser $c_w$ bei 40°C verwendet. Jedoch ist auch $c_w$ temperaturabhängig
und schwankt zwischen 0 und 100°C um ca. \SI{0.019}{\joule\per\gram\per\kelvin} [2].
Dies entspricht zwar nur einer Abweichung von \SI{0.5}{\percent}, ist jedoch
trotzdem eine mögliche Fehlerquelle.
Außerdem wird in allen Versuchsteilen mit Wasser hoher Temperaturen gearbeitet.
Das hat zur Folge, dass kontinuierlich Wasser verdampft wird und die vorher 
gewogenen Wassermassen nicht mehr exakt sind. Außerdem geht mit dem Wasserdampf 
nicht nur Wasser, sondern auch Wärmeenergie verloren. 

Im zweiten Versuchsteil wurden die Molwärmen von Aluminium und Kupfer bestimmt.
Dabei weicht $C_V$ für Aluminium nur um \SI{0.87}{\percent} von $3R$ ab. Demnach
scheint das Dulong-Petitsche Gesetz eine gute Nährung für die Molwärme
von Aluminium zu sein, obwohl dieses ein eher leichtes Element ist. Allerdings 
erfolgte bei Aluminium nur eine Messung, weswegen diese Messung nicht 
repräsentativ ist. 
Die ermittelte Molwärme $C_V$ für Kupfer weicht um \SI{28.91}{\percent} von $3R$ 
ab. Hier scheint die Nährung durch das Dulong-Petitsche Gesetz also nicht mehr 
hinreichend zu sein. Dies ist mit der niedrigeren Atommasse zu erklären. Es muss 
also für Kupfer eine Quantenmechanische Betrachtung erfolgen.
Allerdings wurde während des Wärmeaustausches der Probekörper nicht vollständig
von Wasser bedeckt, was zu dieser hohen Abweichung beigetragen haben kann. 

Zusammenfassend lässt sich sagen, dass das Dunlong-Petitsche Gesetz bei Stoffen
mit hoher Atommasse schon bei recht niedrigen Temperaturen eine gute Nährung 
darstellt. Für Stoffe mit niedriger Atommasse muss jedoch eine 
Quantenmechanische Betrachtung erfolgen.