\section{Diskussion}
\label{sec:Diskussion}

Die experimentell bestimmten Werte der Spaltbreiten und des Spaltabstandes
weichen alle sehr von ihren jeweiligen tatsächlichen Werten ab.
Werden nur die Messwerte betrachtet, so sind die Beugungsbilder so,
wie sie auch zu erwarten sind. Allerdings sind die Funktionen,
auf denen die Fits beruhen sehr komplex und damit schwierig zu fitten,
sodass die Parameter schnell sehr große Fehler besitzen und ungenau bestimmbar
sind. Bei den Messwerten des Doppelspaltes ist der Fit auch dementsprechend
ungenau, da die Messswerte nicht auf einer gut bestimmbaren Funktion liegen.
