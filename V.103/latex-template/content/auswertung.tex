\section{Auswertung}
\label{sec:Auswertung}

\subsection{Bestimmung der Elasitzitätsmodule zweier einseitig eingespannter Stäbe}

\subsubsection{Stab mit quadratischem Querschnitt}

Bei der Messung des Stabes mit rechteckigem Querschnitt, einer Länge 
$L_1= \SI{60}{\centi\meter}$ und einem Gewicht von $m_1 = \SI{464.3}{\gram}$
ergaben sich die Messwerte in Tabelle \ref{tab:Messdaten1}. Dabei ergibt sich 
die Durchbiegung $D$ durch die Differenz von $b_1$ und $b_2$. Der 
Linearisierungsterm ergibt sich mit $Lx²-\frac{x³}{3}$

\begin{table}
\centering
\caption{Biegung eines Stabes mit rechteckigem Querschnitt}
\label{tab:Messdaten1}
\sisetup{table-format=2.1}
\begin{tabular}{c c c c c}
\toprule
$x \,/\, \si{\centi\meter}$ & $b_1 \,/\, \si{\milli\meter}$ & 
$b_2 \,/\, \si{\milli\meter}$ & $D \,/\, \si{\milli\meter}$ &
$Lx²-\frac{x³}{3} \,/\, \SI{e5}{\milli\meter³}$\\
\midrule
 2.5 & -0.005 & 0.060  & 0.065 &   3.323\\
 5.5 & -0.270 & -0.215 & 0.055 &  15.780\\
 8.5 & -0.380 & -0.280 & 0.100 &  36.968\\
11.0 & -0.435 & -0.300 & 0.135 &  60.903\\
13.0 & -0.480 & -0.305 & 0.175 &  83.937\\
15.0 & -0.500 & -0.310 & 0.190 & 110.250\\
17.0 & -0.540 & -0.265 & 0.275 & 139.683\\
19.0 & -0.580 & -0.265 & 0.315 & 172.077\\
21.0 & -0.620 & -0.240 & 0.380 & 207.270\\
23.0 & -0.620 & -0.225 & 0.395 & 245.103\\
25.0 & -0.720 & -0.220 & 0.500 & 285.417\\
27.0 & -0.810 & -0.235 & 0.575 & 328.050\\
29.0 & -0.850 & -0.220 & 0.630 & 372.843\\
31.0 & -0.900 & -0.250 & 0.650 & 419.637\\
33.0 & -0.990 & -0.190 & 0.800 & 468.270\\
35.0 & -1.025 & -0.130 & 0.895 & 518.583\\
37.0 & -1.040 & -0.080 & 0.960 & 570.417\\
39.0 & -1.050 & -0.070 & 0.980 & 623.610\\
41.0 & -1.080 & -0.030 & 1.050 & 678.003\\
42.0 & -1.100 & -0.020 & 1.080 & 705.600\\
43.0 & -1.100 &  0.005 & 1.105 & 733.437\\
44.0 & -1.100 &  0.030 & 1.130 & 761.493\\
45.0 & -1.090 &  0.055 & 1.145 & 789.750\\
46.0 & -1.090 &  0.120 & 1.210 & 818.187\\
47.0 & -1.090 &  0.160 & 1.250 & 846.783\\
48.0 & -1.100 &  0.160 & 1.260 & 875.520\\
49.0 & -1.100 &  0.190 & 1.290 & 904.377\\
49.5 & -1.110 &  0.220 & 1.330 & 918.843\\
\bottomrule
\end{tabular}
\end{table}

Nun wird die Durchbiegung $D$ gegen $Lx²-\frac{x³}{3}$ graphisch aufgetragen
und eine lineare Regressions mittels Python und matplotlib durchgeführt.
Der resultierende Graph ist in Abbildung \ref{fig:plot1} dargestellt.

\begin{figure}
  \centering
  \includegraphics[scale=0.8]{content/plot1.pdf}
  \caption{Durchbiegung mit Regression}
  \label{fig:plot1}
\end{figure}

Die Regression wird mit $D(x) = a_1\cdot x + c_1$ durchgeführt. Dabei ergeben 
sich die Parameter zu: 

\begin{align*}
a_1 &= \SI{0.0140+-0.0002}{1\per\meter²}\\
c_1 &= \SI{0.0752+-0.01364}{\milli\meter}
\end{align*}

Nach Formel \eqref{eqn:Biegung} ergibt sich die Formel: 

\begin{equation*}
E = \frac{F_\text{G}}{2a_1\symbf{I}}
\end{equation*}

Hierbei ist $F_\text{G} = m_\text{g}\cdot g$ mit $m_\text{g} = \SI{0.528}{\kilo\gram}$
und $g = \SI{9.81}{\meter\per\second\squared}$. Das Flächenträgheitsmoment
ergibt sich nach Formel \eqref{eqn:Quadratisch} zu $\frac{1}{12}\SI{e-8}{\meter⁴}$. 
Der Elastizitätsmodul ist also gegeben durch: 

\begin{equation*}
E = \SI{2.220+-0.032e5}{\newton\per\milli\meter²}
\end{equation*}

Der statistische Fehler ergibt sich mit der Gaußschen Fehlerfortpflanzung zu: 

\begin{equation*}
\Delta E = \frac{\partial E}{\partial m_1}\cdot \Delta m_1 
= \frac{-F_\text{G}}{2\symbf{I}m²} \cdot \Delta m_1
\end{equation*}

%Es scheint sich demnach bei dem Material um Stahl zu handeln, dessen 
%Elasitzitätsmodul bei $E=\SI{220000}{\newton\per\milli\meter²}$ [2] liegt.

\subsubsection{Stab mit zylindrischem Querschnitt}

Das Verfahren läuft analog mit dem runden Stab mit $L = \SI{55.2}{\centi\meter}$.
Die entsprechenden Messdaten finden sich in Tabelle \ref{tab:Messdaten2}. 

\begin{table}
\centering
\caption{Biegung eines Stabes mit zylindrischem Querschnitt}
\label{tab:Messdaten2}
\sisetup{table-format=2.1}
\begin{tabular}{c c c c c}
\toprule
$x \,/\, \si{\centi\meter}$ & $b_1 \,/\, \si{\milli\meter}$ & 
$b_2 \,/\, \si{\milli\meter}$ & $D \,/\, \si{\milli\meter}$ &
$Lx²-\frac{x³}{3} \,/\, \SI{e5}{\milli\meter³}$\\
\midrule
 2.5 &  0.000 & 0.170 & 0.170 &   3.398\\
 5.5 & -0.110 & 0.050 & 0.160 &  16.143\\
 8.5 & -0.310 & 0.020 & 0.330 &  37.835\\
11.0 & -0.395 & 0.030 & 0.425 &  62.355\\
13.0 & -0.455 & 0.065 & 0.520 &  85.965\\
15.0 & -0.510 & 0.013 & 0.523 & 112.950\\
17.0 & -0.570 & 0.270 & 0.840 & 143.151\\
19.0 & -0.600 & 0.400 & 1.000 & 176.409\\
21.0 & -0.645 & 0.560 & 1.205 & 212.562\\
23.0 & -0.710 & 0.775 & 1.485 & 251.451\\
25.0 & -0.680 & 0.985 & 1.665 & 292.917\\
27.0 & -0.650 & 1.150 & 1.800 & 336.798\\
29.0 & -0.630 & 1.565 & 2.195 & 382.935\\
31.0 & -0.600 & 1.700 & 2.300 & 431.169\\
33.0 & -0.560 & 2.000 & 2.560 & 481.338\\
35.0 & -0.560 & 2.260 & 2.820 & 533.283\\
37.0 & -0.555 & 2.460 & 3.015 & 586.845\\
39.0 & -0.550 & 2.820 & 3.370 & 641.862\\
41.0 & -0.550 & 3.050 & 3.600 & 698.175\\
42.0 & -0.550 & 3.200 & 3.750 & 726.768\\
43.0 & -0.550 & 3.385 & 4.935 & 755.625\\
44.0 & -0.550 & 3.500 & 4.050 & 784.725\\
45.0 & -0.540 & 3.630 & 4.170 & 814.050\\
46.0 & -0.520 & 3.780 & 4.300 & 843.579\\
47.0 & -0.500 & 3.930 & 4.430 & 873.291\\
48.0 & -0.480 & 4.030 & 4.510 & 903.168\\
\bottomrule
\end{tabular}
\end{table}

Trägt man nun erneut $D$ gegen $L\cdot x²-\frac{x³}{3}$ auf und führt eine
Ausgleichsrechnung mittels matplotlib durch, ergibt sich Abbildung \ref{fig:plot2}. 

\begin{figure}
  \centering
  \includegraphics[scale=0.8]{content/plot2.pdf}
  \caption{Durchbiegung mit Regression}
  \label{fig:plot2}
\end{figure}

Die Regressionsparamter ergeben sich dabei zu: 

\begin{align*}
a_2 &= \SI{0.0497+-0.0004}{1\per\meter²}\\ 
c_2 &= \SI{0.1362+-0.0232}{\milli\meter}
\end{align*}

Bei diesem Stab ist das Flächenträgheitsmoment nach Formel \eqref{eqn:Rund}
gegeben durch $\frac{\pi}{64}\si{\centi\meter⁴}$.
Schließlich ergibt sich für den Elasitzitätsmodul $E$ wie oben mit Formel 
\eqref{eqn:Biegung} und durch das Flächenträgheitsmoment:

\begin{equation*}
E = \SI{1.037+-0.008e5}{\newton\per\milli\meter²}
\end{equation*}

Dabei betrug die Masse des angehängten Gewichtes $m_2 = \SI{0.516}{\kilo\gram}$.

\subsection{Bestimmung des Elasitzitätsmodul eines beidseitig aufgelegten Stabes}

Die aufgenommenen Messdaten eines beidseitig aufgelegten eckigen Stabes 
mit Länge $L = \SI{60.3}{\centi\meter}$ finden sich in Tabelle \ref{tab:Messdaten3}. 
Die Durchbiegung wird durch $D = b_1 - b_2$ berechnet. Außerdem wird für die
rechte Seite von dem Gewicht der Linearisierungsterm $3L²x-4x³$ und für die
linke Seite $4x³-12Lx² + 9L²x-L³$ bestimmt. 

\begin{table}
\centering
\caption{Biegung eines beidseitig aufgelegten Stabes}
\label{tab:Messdaten3}
\sisetup{table-format=2.1}
\begin{tabular}{c c c c c}
\toprule
$x \,/\, \si{\centi\meter}$ & $b_1 \,/\, \si{\milli\meter}$ & 
$b_2 \,/\, \si{\milli\meter}$ & $D \,/\, \si{\milli\meter}$ &
$3L²x-4x³ \,/\, \SI{e5}{\milli\meter³}$\\
\midrule
 2.5 & 0.005 & 0.130 & 0.125 &  181.180\\
 5.5 & 0.015 & 0.165 & 0.150 &  393.315\\
 8.5 & 0.060 & 0.280 & 0.220 &  593.570\\
11.0 & 0.130 & 0.390 & 0.260 &  746.700\\
13.0 & 0.110 & 0.450 & 0.340 &  857.503\\
15.0 & 0.180 & 0.570 & 0.390 &  955.827\\
17.0 & 0.220 & 0.650 & 0.430 & 1039.751\\
18.0 & 0.195 & 0.655 & 0.460 & 1075.712\\
19.0 & 0.225 & 0.660 & 0.435 & 1107.354\\
20.0 & 0.240 & 0.730 & 0.490 & 1134.436\\
21.0 & 0.270 & 0.740 & 0.470 & 1156.718\\
22.0 & 0.245 & 0.755 & 0.510 & 1173.960\\
23.0 & 0.250 & 0.765 & 0.515 & 1185.921\\
24.0 & 0.255 & 0.790 & 0.535 & 1192.363\\
25.0 & 0.260 & 0.800 & 0.540 & 1193.045\\
26.0 & 0.310 & 0.810 & 0.500 & 1187.727\\
27.0 & 0.320 & 0.950 & 0.630 & 1176.169\\
\midrule
$x \,/\, \si{\centi\meter}$ & $b_1 \,/\, \si{\milli\meter}$ & 
$b_2 \,/\, \si{\milli\meter}$ & $D \,/\, \si{\milli\meter}$ &
$4x³-12Lx²+9L²x-L³ \,/\, \SI{e5}{\milli\meter³}$\\
\midrule
28.0 & 0.300 & 0.840 & 0.540 & 2175.441\\
29.0 & 0.330 & 0.850 & 0.520 & 2187.717\\
30.0 & 0.335 & 0.850 & 0.515 & 2192.481\\
31.0 & 0.355 & 0.850 & 0.495 & 2189.973\\
32.0 & 0.350 & 0.890 & 0.540 & 2180.433\\
33.0 & 0.365 & 0.855 & 0.490 & 2164.101\\
34.0 & 0.355 & 0.850 & 0.495 & 2141.217\\
35.0 & 0.365 & 0.870 & 0.505 & 2112.021\\
37.0 & 0.380 & 0.840 & 0.460 & 2035.653\\
39.0 & 0.385 & 0.840 & 0.455 & 1936.918\\
41.0 & 0.400 & 0.830 & 0.430 & 1817.734\\
43.0 & 0.415 & 0.780 & 0.365 & 1680.022\\
45.0 & 0.415 & 0.750 & 0.335 & 1525.702\\
48.0 & 0.420 & 0.650 & 0.230 & 1267.283\\
49.5 & 0.420 & 0.600 & 0.180 & 1127.705\\
\bottomrule
\end{tabular}
\end{table}

Für beide Seiten wie Durchbiegung $D$ graphisch gegen den jeweiligen
Linearisierungsterm aufgetragen und jeweils eine lineare Regression
durchgeführt. Dabei ergeben sich die Abbildungen \ref{fig:plot3} und 
\ref{fig:plot4}. 

\begin{figure}
  \centering
  \includegraphics[scale=0.8]{content/plot3.pdf}
  \caption{Messdaten und Regression der rechten Stabseite}
  \label{fig:plot3}
\end{figure}

\begin{figure}
  \centering
  \includegraphics[scale=0.8]{content/plot4.pdf}
  \caption{Messdaten und Regression der linken Stabseite}
  \label{fig:plot4}
\end{figure}

Die Regressionsparamter der Regression für die rechte Seite des Stabes
lauten:

\begin{align*}
a_3 &= \SI{0.00449+-0.00004}{1\per\meter²}\\
c_3 &= \SI{-0.0198+-0.0377}{\milli\meter}
\end{align*}

Und für die linke Seite des Stabes ergibt sich: 

\begin{align*}
a_4 &= \SI{0.00308+-0.00001}{1\per\meter²}\\
c_4 &= \SI{-0.1534+-0.0275}{\milli\meter}
\end{align*}

Gemäß der Formel \eqref{eqn:Beidseitig} gilt für die beiden Elasitzitätsmodule, rechts und links, 
der Zusammenhang: 

\begin{equation*}
E = \frac{F_\text{G}}{48a_\text{i}\symbf{I}} \text{ mit i = 3, 4}
\end{equation*}

Hier ist $F_\text{G} = m\cdot g$ mit $m = \SI{1.2033}{\kilo\gram}$ und 
$g = \SI{9.81}{\meter\per\second²}$.
Mit \eqref{eqn:Quadratisch} folgt für $\symbf{I} = \frac{1}{12}\SI{e-8}{\meter⁴}$. Damit 
ergibt sich für die Elasitzitätsmodule: 

\begin{align*}
E_\text{R} &= \SI{6.57+-0.06e4}{\newton\per\milli\meter²}\\
E_\text{L} &= \SI{9.581+-0.031e4}{\newton\per\milli\meter²}
\end{align*}

Bildet man aus diesen Werten den Mittelwert ergibt sich schließlich: 

\begin{equation*}
E_\text{M} = \SI{8.0755+-1.5055e4}{\newton\per\milli\meter²}
\end{equation*}

wobei der Fehler mit 

\begin{equation*}
\Delta E_\text{M} = \sqrt{(\Delta E_\text{R})²+(\Delta E_\text{L})²}
\end{equation*}

berechnet wird.