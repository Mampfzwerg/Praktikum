\section{Diskussion}
\label{sec:Diskussion}

Die Werte der Elastizitätsmodule ergeben sich zu

\begin{align*}
E_Q &= \SI{2.220+-0.032e5}{\newton\per\milli\meter²}\\
E_K &= \SI{1.037+-0.008e5}{\newton\per\milli\meter²}
\end{align*}

Es ist zu erkennen, dass die Bestimmung dieser Elastizitätsmodule gut 
funktioniert, was sich dadurch äußert, dass der statische Fehler
beider Elastizitätsmodule etwa bei $\SI{1}{\percent}$ liegt. 
Systematische Abweichung sind hier demnach nicht zu erkennen. Somit 
kann die theoretische Formel \eqref{eqn:Biegung} zur Beschreibung 
der Biegung elastischer Stäbe (bei kleinen Auslängen gegenüber der
Stablänge) als bestätigt angesehen werden.

Bei der Bestimmung des Elastizitätsmodul eines beidseitig aufliegenden
Stabes mit quadratischem Querschnitt erhält man folgenden Mittelwert: 

\begin{equation*}
E_M = \SI{8.0755+-1.5055e4}{\newton\per\milli\meter²}
\end{equation*}

Es ist zu erkennen, dass der statische Fehler des Mittelwertes mit 
$\SI{18.64}{\percent}$ recht groß ist. Dies ist mit dem großen Unterschied
der beiden Elastizitätsmodule der rechten und der linken Seite zu 
erklären. Ein möglicher Grund dafür ist, dass das angehängte Gewicht
nicht exakt in der Mitte des Stabes plaziert worden ist. Somit resultiert
eine leicht asymmetrische Biegung des Stabes. Daraus folgen auch für beide
Seiten des Stabes unterschiedliche Elastizitätsmodule, da in der Herleitung
von einer perfekt symmetrischen Durchbiegung ausgegangen wird. 

Desweiteren kann man Folgerungen treffen, aus welchem Material die 
Stäbe bestehen. Der einseitig eingespannte quadratische Stab scheint aus
Stahl zu bestehen. Nicht rostender Stahl hat ein Elastizitätsmodul von 
$\SI{220000}{\newton\per\milli\meter²}$ [2] und eine Dichte von 
$\SI{7.85}{\gram\per\centi\meter³}$ [3], was gut zu dem ermittelten 
Wert passt. Die ermittelte Dichte des Stabes passt mit 
$\SI{7.74}{\gram\per\centi\meter³}$ ebenfalls zu diesen Literaturwerten.
\\
Der einseitig eingespannte runde Stab besteht anhand des Elastizitätsmodul
entweder aus Messing oder Kuper. Der Literaturwert des Elastizitätsmoduls
von Messing variiert von $\num{78}$ bis $ \SI{123}{\giga\pascal}$ [4] und fasst
damit den im Experiment ermittelten Wert mit ein. Das Gleiche gilt für 
Kupfer, dessen Literaturwert des Elastizitätsmoduls von
$\num{100}$ bis $\SI{130}{\giga\pascal}$ variiert. Die ermittelte Dichte des 
Stabes liegt bei $\SI{8.31}{\gram\per\centi\meter³}$, was besser zu dem Wert 
der Dichte von Messing mit $\SI{8.73}{\gram\per\centi\meter³}$ [3] passt. 
Demnach scheint dieser Stab aus Messing zu bestehen. 
\\
Zuletzt wird der beidseitig aufliegende Stab betrachtet. Die Bestimmung 
des Materials anhand des Elastizitätsmoduls erweist sich als schwierig, 
da der statische Fehler sehr groß ist und demnach viele Materialien in
Betracht gezogen werden müssen. Die Dichte von $\SI{8.33}{\gram\per\centi\meter³}$
deutet allerdings erneut auf Messing hin. Das ermittelte Elastizitätsmodul
passt allerdings nicht zu dieser Deutung. Demnach scheint dieses falsch 
ermittelt worden sein, wie bereits vermutet. 