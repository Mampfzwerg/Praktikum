\input{header.tex}

\subject{Nr. 103}
\title{Biegung elastischer Stäbe}
\date{%
  Durchführung: 27.11.2018
  \hspace{3em}
  Abgabe: 04.12.2018
}

\begin{document}

\maketitle
\thispagestyle{empty}
\tableofcontents
\newpage

\section{Theorie}
\label{sec:Theorie}

Ziel des Versuches ist es, gedämpfte und erzwungene Schwingungen eines
RLC-Schwingkreises zu untersuchen und den Dämpfungswiderstand und die 
Frequenzabhängigkeit der Phasenverschiebung zwischen Kondensator- und 
Generatorspannung zu bestimmen. 
Ein solcher Schwingkreis zeichnet sich aus durch eine Kapazität $C$, 
einer Induktivität $L$ und einem einem Widerstand $R$. Bei einer 
Schwingung wird beständig Energie zwischen dem B-Feld der Spule und
dem E-Feld des Kondensators ausgetauscht. Der Widerstand $R$ sorgt
dafür, dass die Energie des Systems fortlaufend abnimmt und so die 
Amplituden immer kleiner werden, bis sie irgendwann gänzlich abgeklungen 
sind. 
Mit Hilfe des zweiten Kirchhoffschen Gesetztes kann man eine Gleichung 
für den Schaltkreis aufstellen: 

\begin{equation*}
U_R(t) + U_C(t) + U_L(t) = 0 
\end{equation*}

Durch Umformungen kann eine DGL für das System aufgestellt werden: 

\begin{equation}
L\frac{\symup{d}²I}{\symup{d}t²}+\frac{R}{L} \frac{\symup{d}I}{\symup{d}t}+\frac{I}{LC} = 0
\end{equation}

Eine Lösung der DGL ist: 

\begin{equation}
I(t) = e^{-t\frac{R}{2L}}\left(I_1 e^{it\sqrt{\frac{1}{LC}-\frac{R²}{4L²}}}+I_2 e^{-it\sqrt{\frac{1}{LC}-\frac{R²}{4L²}}} \right).
\label{eqn:Loesung}
\end{equation}

Es gibt drei verschiedene Fälle, die betrachtet werden müssen. 
Der erste Fall ist, dass $\frac{1}{LC} > \frac{R²}{4L²}$. Die Wurzel im Exponenten
bleibt reell und \eqref{eqn:Loesung} geht über in: 

\begin{equation*}
I(t) = A_0 e^{-\frac{R}{2L}t}\cdot \cos\left({2\pi ft+\eta}\right).
\end{equation*}

Hierbei ist $\eta$ eine Phasenverschiebung. Dieser Fall wird auch als 
Schwingfall bezeichnet. 
Wenn nun gilt

\begin{align*}
\frac{1}{LC} &= \frac{R²}{4L²}\\
\leftrightarrow R &= 2\sqrt{\frac{L}{C}},
\end{align*}

spricht man vom aperiodischen Grenzfall. Bei diesem Fall geht $I(t)$ ohne 
Überschwingung am schnellsten gegen null. 

Wird der RLC-Kreis nun von außen durch eine Wechselspannung zum Schwingen 
angeregt, so spricht man von einer erzwungenen Schwingung. In diesem Fall
liegt eine Inhomogenität in der DGL vor. Durch Lösen der resultierenden 
DGL erhält man schließlich: 

\begin{equation}
U_C(\omega) = \frac{U_0}{\sqrt{(1-LC\omega²)²+\omega²R²C²}}
\end{equation}

Diese Lösung hat ein Maximum bei $f = f_\text{res}$, das Resonanzüberhöhung
genannt wird. Die Breite $\symup{\Delta}f$ kann mit

\begin{equation}
\symup{\Delta} f = |f_1 - f_2|
\end{equation}

bestimmt werden. Hierbei gilt: 

\begin{equation}
f_\text{res} = \frac{1}{2\pi}\sqrt{\frac{1}{LC}-\frac{R²}{2L²}}
\end{equation}

und 

\begin{equation}
f_\text{1,2} = \pm \frac{1}{2\pi} \frac{R}{2L}+\sqrt{\frac{R²}{4L²}+\frac{1}{LC}}.
\end{equation}

Außerdem weist die Kondensatorspannung $U_C(t)$ eine Phasenverschiebung gegenüber
der Erregerspannung auf. Diese kann mit 

\begin{equation}
\tan({\phi(\omega)})=\frac{-\omega RC}{1-LC\omega²}
\end{equation}

bestimmt werden. 
\section{Durchführung}
\label{sec:Durchführung}

Für die Bestimmung des Schubmoduls $G$ nach der dynamischen Methode wird die in Abbildung ... abgebildete
Versuchsapparatur verwendet. 

In dieser ist eine Kugel an einem Torsionsdraht aufgehängt. Der Draht soll durch eine Auslenkung
mit dem Justierrad zum schwingen gebracht werden. Zum unterbinden andersförmiger Bewegungen
kann die Kugel durch eine Dämpfung in Ruhe versetzt werden.

Zur Messung der Periodendauer wird nach Abbildung ... ein Lichtstrahl emittiert, 
der durch einen Doppelspalt und eine Sammellinse auf einen am Draht befestigten Spiegel fällt. 
Der Spiegel dreht sich somit zusammen mit der Kugel und dem reflektierten Lichtstrahl, 
welcher an einem Punkt auf eine Fotodiode trifft.

Das elektrische Signal wird dabei zur Zeitmessung genutzt. Das erste Signal beginnt die Messung,
das zweite muss füreine volle Periode ignoriert werden, was durch eine sogenannte Flip-Flop-Kippstufe realisiert wird.
Das dritte Signal beendet dann die Messung und das vierte setzt die Zähluhr zurück. 
Zu Anfang soll dazu der Draht mittels des Justierrades so justiert werden, dass der Lichtstrahl
auf die, die Diode umgebende, Mattscheibe fällt. Auch sollen die Abstände innerhalb der Beleuchtuns-
apparatur so variiert werden, dass das Beugungsbild möglichst scharf auf der Mattscheibe abgebildet wird.\\

Zusätzlich wird eine Messung zur Bestimmung des magnetischen Dipolmoments $\vec{m}$ eines Permanentmagneten
durchgeführt. Während bei der Messung des Schubmoduls $G$ die Dipolachse der Kugel anhand einer Markierung
parallel zum Draht ausgerichtet werden sollte, um den Einfluss des Erdmagnetfeldes aufzuheben, wird
dieses hier senkrecht zum Draht ausgerichtet. Außerdem wird um den Magneten durch ein Helmholtzspulenpaar
ein homogenes Magnetfeld aufgebaut. Wieder wird die Periodendauer gemessen, diesmal jeweils für verschiedene
magnetische Flussdichten. Außerdem darf der Schwingungswinkel bei dieser Messung nur klein sein, da somit eine 
vereinfachende Winkelnäherung in der Herleitung verwendet werden kann.\\

Außerdem sollen mit einer Mikrometerschraube der Durchmesser und mit einem Maßband die Länge des Drahtes gemessen werden.




\section{Auswertung}
\label{sec:Auswertung}

\subsection{Zeitabhängigkeit der Amplitude}

Die gemessenen Maxima bei einer gedämpften Schwingung sind 
in Tabelle \ref{tab:Messdaten1} zu sehen. 

\begin{table}
\centering
\caption{Messdaten der Maxima der Amplitude}
\label{tab:Messdaten1}
\sisetup{table-format=2.1}
\begin{tabular}{c c}
\toprule
$U \,/\, \si{\volt}$ & $s \,/\, \si{\micro\second}$\\
\midrule
1,74 &   0,0\\
1,44 &  29,6\\
1,20 &  58,8\\
1,10 &  88,2\\
0,84 & 117,2\\
0,68 & 147,2\\
0,56 & 177,2\\
0,46 & 206,2\\
0,38 & 235,2\\
0,30 & 265,2\\
0,22 & 294,2\\ 
0,16 & 324,2\\
0,14 & 353,2\\
0,10 & 382,2\\
0,06 & 412,2\\
0,02 & 441,2\\
0,00 & 471,2\\
\bottomrule
\end{tabular}
\end{table} 

Die Ausgleichsrechnung wird mit der Funktion 

\begin{equation*}
A = A_0 \cdot e^{-2\pi\mu t}
\end{equation*}

durchgeführt. Das Ergebnis ist in Abbildung \ref{fig:gedämpft} zu sehen. 

\begin{figure}
  \centering
  \includegraphics[scale=0.8]{content/plot1.pdf}
  \caption{Exponentielle Regression der Amplitude}
  \label{fig:gedämpft}
\end{figure}

Mittels python ergeben sich die Regressionsparamter zu: 

\begin{align*}
A_0 &= \SI{1.785+-0.036}{\volt},\\
\mu &= \SI{1068.421+-34.320}{\per\second}.
\end{align*}

Mit Formel... lässt sich nun der effiktive Widerstand berechnen.

\begin{equation*}
R_\text{eff} = 4\pi L\mu = \SI{136+-4}{\ohm}
\end{equation*}

Der Fehler ergibt sich dabei durch die Gaußsche Fehlerfortpflanzung zu: 

\begin{equation*}
\symup{\Delta} R_\text{eff} = \sqrt{\left(\frac{\symup{d}R_\text{eff}}{\symup{d}L}\right)²\cdot (\symup{\Delta}L)² +
\left(\frac{\symup{d}R_\text{eff}}{\symup{d}\mu}\right)²\cdot (\symup{\Delta}\mu)²}.
\end{equation*}

Weiterhin wird die Abklingdauer mit Formel... berechnet und 
es ergibt sich:

\begin{equation*}
T_\text{ex} = \frac{1}{2\pi\mu} = \SI{0.149+-0.005e-3}{\second}.
\end{equation*}

Der Fehler ergibt sich hierbei zu: 

\begin{equation*}
\symup{\Delta} T_\text{ex} = \sqrt{\left(\frac{\symup{d}T_\text{ex}}{\symup{d}\mu}\right)²\cdot (\symup{\Delta}\mu)²}.
\end{equation*}

\subsection{Bestimmung des Dämpfungswiderstandes}

Hier wurde der aperiodische Grenzfall untersucht. Dabei wurde der
Dämpfungswiderstand zu 

\begin{equation*}
R_\text{ap} = \SI{3520+-50}{\ohm}
\end{equation*}

bestimmt.
Der theoretische Wert von $R_\text{ap}$ kann mit Formel \eqref{eqn:apth} bestimmt 
werden: 

\begin{equation*}
R_\text{ap,theo} = \SI{4390+-9}{\ohm}.
\end{equation*}

Der Fehler berechnet sich über die Gaußsche Fehlerfortpflanzung: 

\begin{equation*}
\symup{\Delta} R_\text{ap,theo} = \sqrt{\left(\frac{\symup{d}R_\text{ap}}{\symup{d}L}\right)²\cdot (\symup{\Delta}L)² +
\left(\frac{\symup{d}R_\text{ap}}{\symup{d}C}\right)²\cdot (\symup{\Delta}C)²}.
\end{equation*}

\subsection{Frequenzabhängigkeit der Kondensatorspannung}
\section{Diskussion}
\label{sec:Diskussion}

Weitesgehend zeigt sich in allen Versuchsteilen eine sehr hohe Übereinstimmung zwischen den gemessenen
und theoretisch berechneten Werten.

\printbibliography{}

\section{Literaturverzeichnis}

[1]: \ TU Dortmund. \textit{Versuchsanleitung zu Versuch 103: Biegung elastischer Stäbe.}\newline
[2]: \ \url{https://vergleichsspannung.de/glossar/e-modul/}
\textit{(Werte entnommen am 01.12.2018)}\newline
[3]: \ Wikibooks. \url{https://de.wikibooks.org/wiki/Tabellensammlung_Chemie/_Dichte_fester_Stoffe}
\textit{(Werte entnommen am 01.12.2018)}\newline
[4]: \ Wikipedia. \url{https://de.wikipedia.org/wiki/Elastizit%C3%A4tsmodul}
\textit{(Werte entnommen am 01.12.2018)}\newline

\end{document}
