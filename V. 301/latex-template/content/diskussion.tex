\section{Diskussion}
\label{sec:Diskussion}

Zunächst wird die Leerlaufspannung $U_0$ direkt gemessen, wobei sich ein 
systematischer Fehler von \SI{933}{\nano\volt} ergibt. Dieser ist jedoch 
so gering, dass keines der verwendeten Messinstrumente dies hätte regestrieren 
können. 

Im weiteren Verlauf des Experiments werden die Leerlaufspannung und der 
Innenwiderstand der Monozelle erst ohne und dann mit Gegenspannung bestimmt. 
Man erkennt dabei, dass die Werte bei beiden Methoden sehr nahe beieinander 
liegen. So weichen die Werte für den Innenwiderstand um \SI{0.422}{\ohm} 
und die der Leerlaufspannung um \SI{0.021}{\volt} voneinander ab. 
Die Abweichung des Innenwiderstandes kann damit erklärt werden, dass bei 
der Gegenspannungsmethode eine weitere Spannungsquelle verwendet wird, die 
auch mit einem Innenwiderstand behaftet ist. 

Alle errechneten Fehler sind recht gering. Es ergibt sich ein systematischer Fehler, 
wenn das Voltmeter im Punkt H angelegt wird, wie in Abbildung \ref{fig:Messschaltung}
angedeutet. 

Bei der Leistungs-Widerstandskennlinie ist kein systematischer Fehler 
zu erkennen. Die Abweichungen zwischen Theoriekurve und Messdaten sind
auf Messungenauigkeiten zurückzuführen. 