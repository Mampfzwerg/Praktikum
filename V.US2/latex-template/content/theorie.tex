\section{Theorie}
\label{sec:Theorie}

Ziel des Versuches ist das kennenlernen und anwenden der Scanverfahren
in der Ultraschallechographie. Im Zuge dessen werden Störstellen in einem 
Acrylblock vermessen und ein Modell eines menschlichen Herzens beobachtet.

Die Frequenzen des Ultraschalls von $\SI{20}{\kilo\hertz}$ bis
$\SI{1}{\giga\hertz}$ liegen über dem Hörbereich des Menschen
von $\SI{16}{\kilo\hertz}$ bis $\SI{20}{\kilo\hertz}$.

Schall beschreibt eine longitudinale Druckwelle

\begin{equation}
    p(x, t) = p_0 + v_0 Z \text{cos} \left(\omega t - k x \right)
\end{equation}

mit der akustischen Impedanz $Z = c\rho$ , die von der Dichte des
durchstrahlten Materials und dessen Schallgeschwindigkeit $c$ abhängt.
Auch die Schallgeschwindigkeit ist materialabhängig.
In Gasen und Flüssigkeiten breiten sich Longitudinalwellen mit der 
von der Kompressibilität $\kappa$ und Dichte $\rho$ abhängiger
Geschwindigkeit

\begin{equation}
    c_\text{Fl} = \sqrt{\frac{1}{\kappa \rho}}
\end{equation}

aus. In Festkörpern breiten sich Longitudinalwellen
mit der vom Elastizitätsmodul $E$ abhängigen Schallgeschwindigkeit 

\begin{equation}
    c_\text{Fe} = \sqrt{\frac{E}{\rho}}
\end{equation}

aus. Auch Transversalwellen mit anderer Schallgeschwindigkeit sind hier
möglich. Allgemein hängen Schallgeschwindigkeiten von der Wellenrichtung
ab.

Ein Teil des Schalles wird bei seiner Ausbreitung absorbiert, somit
gilt für die Intensität:

\begin{equation}
    I(x) = I_0 \cdot \text{e}^{\alpha x}
\end{equation}

$\alpha$ beschreibt den Absorptionskoeffizienten der Schallamplitude

Zudem werden Schallwellen an Grenzflächen reflektiert. Der Reflektionskoeffizient

\begin{equation}
    R = \left( \frac{Z_1 - Z_2}{Z_1 + Z_2} \right)^2
\end{equation}

gibt das Verhältnis von einfallender und reflektierter Wellen,

der Transmissionskoeffizient $T = R - 1$
jenes einfallender und transmittierter Wellen an. \\

Eine Methode zur Ultraschallerzeugung stellt der
piezo-elektrische Effekt dar. Dabei wird ein
(Quarz-)Kristall durch ein wechselndes
elektrisches Feld zur Schwingung angeregt und
emittiert Ultraschallwellen. Besonders groß
ist deren Amplitude, wenn die Anregung die 
Resonanzfrequenz trifft. Umgekehrt wird der
Kristall auch verwendet, um Ultraschall zu detektieren,
dadurch dass dieser zu einer messbaren Schwingung angeregt wird. \\

Verwendung finden bspw. in der Medizin Laufzeitmessungen des Ultraschalls,
aus denen sich auf das zu durchquerende Material schließen lässt.
Beim Durchschallungsverfahren wird Ultraschall ausgesendet nach
Durchquerung einer Materialprobe wieder empfangen.
Liegen im Material Störstellen vor, so wird an diesen
eine geringere Intensität gemessen.
Beim Impuls-Echo-Verfahren liegen Sender und
Empfänger zusammen. Der Schall wird dazu am Ende der Probe
an einer Grenzfläche reflektiert. Mit dieser Methode 
lassen sich die Größen möglicher Störstellen  und bei bekannter
Schallgeschwindigkeit auch deren Höhen 

\begin{equation}
    s = \frac{1}{2} c t
\end{equation}

bestimmen.
Zur Darstellung der Laufzeitdiagramme dienen die Darstellungsarten:

\begin{itemize}
    \item A-Scan (Amplituden Scan): Ein eindimensionales Verfahren, bei dem die 
    Echoamplituden eine Funktion der Laufzeit sind.
    \item B-Scan (Brightness Scan): Ein zweidimensionales Verfahren, bei dem
    durch Bewegung der Sonde ein Bild entsteht, in dem
    die Echoamplituden durch Helligkeitsabstufungen dargestellt sind. 
    \item TM-Scan (Time-Motion Scan): Ein schnelles Verfahren zur Aufnahme
    von Bildfolgen.
\end{itemize}










