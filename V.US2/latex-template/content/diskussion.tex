\section{Diskussion}
\label{sec:Diskussion}

Bei der Untersuchung der beiden kleinsten Fehlstellen ist auffällig, dass das Auflösungsvermögen 
der beiden Fehlstellen mit wachsender Frequenz geringer wird. 
Bei dem Vergleich der bestimmten Durchmesser der Fehlstellen fällt auf, dass die mit dem 
B-Scan bestimmten Werte im Vergleich zu den mit den A-Scan bestimmten Werte meist größer sind. 
Im Mittel ergibt sich eine Differenz von $\SI{0.87}{\milli\meter}$, was bei diesen Größenordnung 
einer recht großen Abweichung entspricht. In Tabelle \ref{tab:Abw} sind die Abweichungen der Ergebnisse des
B-Scans von den Ergebnissen des A-Scans für jedes Loch aufgeführt. 

\begin{table}
\centering
\caption{Abweichungen der Durchmesser des B-Scans von denen des A-Scans.}
\label{tab:Abw}
\sisetup{table-format=2.1}
\begin{tabular}{c c}
\toprule
Stelle & \\
\midrule 
9 & $\SI{6.93}{\percent}$\\
8 & $\SI{1.31}{\percent}$\\
7 & $\SI{6.82}{\percent}$\\
6 & $\SI{6.91}{\percent}$\\
5 & $\SI{31.09}{\percent}$\\
4 & $\SI{65.53}{\percent}$\\
3 & $\SI{69.78}{\percent}$\\
\bottomrule
\end{tabular}
\end{table}


Insgesamt wirkt der A-Scan allerdings zuverlässiger, da schon mit bloßem Auge ersichtlich ist, 
dass die Löcher von drei bis neun zunehmend kleiner werden. Beim B-Scan ist diese Tendenz, 
im Gegensatz zu den Ergebnissen des A-Scans, nicht zu erkennen. 
Desweiteren ist nicht klar, wieviel Laufzeit tatsächlich durch das Acryl erfolgte und wieviel durch 
das Sondenmaterial bzw. das Koppelmittel erfolgte. Außerdem kann auch die Dicke des Koppelmittels 
variiren. 
Im Vergleich mit den Literaturwerten scheint das bestimmte Herzvolumen ein realistischer Wert zu sein. 
