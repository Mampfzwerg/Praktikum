\section{Durchführung}
\label{sec:Durchführung}

Der Versuchsaufbau beinhaltet aus einem Ultraschallechoskop,
Ultraschallsonden verschiedener Frequenzen, einem Computer
zur Datenaufnahme und einem Kippschalter, der zwischen
Durchschallung- und Impuls-Echo-Verfahren wechselt. \\

Zuerst wird ein Acrylblock mit dem A-Scan untersucht.
Die Abmessungen des Blockes werden mithilfe einer 
Schieblehre, darauf die Lage der Bohrungen mit dem
Impuls-Echo-Verfahren, und deren Tiefe mit dem A-Scan
aus verschiedenen Winkeln bestimmt. \\

Das Auflösungsvermögen wird durch die Messung zweier
nebeneinanderliegender Bohrungen untersucht.\\

Zudem wird der Acrylblock mit dem B-Scan untersucht. Dabei ist
auf eine konstante Geschwindigkeit der Sondenbewegung zu achten.\\

Zuletzt wird an einem Herzmodell der TM-Scan durchgeführt.
Dieses wird zu einem Drittel mit Wasser gefüllt und ein
A-Scan der Echolaufzeit durchgeführt.
Dabei soll das Herzvolumen vergrößert und verkleinert werden
und der menschliche Herzschlag simuliert werden.

