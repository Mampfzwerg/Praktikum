\section{Diskussion}
\label{sec:Diskussion}

Während der gesamten Messung kommt es zu nicht auswertbaren
und offensichtlich falschen Werten für die Periodendauern.
Dies hängt damit zusammen, dass das Lichtsignal durch die Diode 
gelegentlich mehrfach oder keinmal gezählt wird. Bei der Messungzur Bestimmung
des magnetischen Moments trat dieser Fehler umso öfter auf, da für kleine 
Winkel die Auslenkung zu schwach sein konnte und der Umkehrpunkt des Lichtstrahls
somit nah an der Diode. 

Des Weiteren wurde die Messung durch kleine Erschütterungen und die nicht völlig
unterbindbare Pendelbewegung der Kugel beeinflusst. 

Wegen dieser Umstände weicht der Wert des gemessenen Schubmoduls um $\SI{79.31}{\percent}$
vom Literatruwert $ G_\text{lit} = \SI{29.3}{\giga\pascal}$ ab.
