\section{Theorie}

Ziel dieses Versuches ist die Bestimmung der Wellenlänge $\lambda$ eines Diodenlasers und
des Brechungsindizes $n$ von Luft mithilfe des Michelson-Interferometers.

Licht ist eine elektromagnetische Welle mit elektromagnetischer Feldstärke

\begin{equation}
    \vec{E}(x, t) = \vec{E_0} \cdot \text{cos}(kx - \omega t - \delta) \; ,
    \label{eqn:ansatz}
\end{equation}

für die das Superpositionsprinzip gilt.

Es besitzt die Wellenlänge $\lambda$, die Wellenzahl $k = \frac{2 \pi}{\lambda}$, die Kreisfrequenz $\omega$,
die Phase $\delta$ und die Intensität

\begin{equation}
    I \propto |\vec{E}|^2 \; .
    \label{eqn:props}
\end{equation}

Die Gesamtintensität an einem Ort, auf den die Wellen $\vec{E_1}$ und $\vec{E_2}$ treffen, ergibt sich 
unter der Bedingung, dass $t_2 - t_1$ groß gegen die Periodendauer $T = \frac{2 \pi}{\omega}$ ist, als

\begin{equation}
    I_\text{ges} = \frac{C}{t_2 - t_1} \int^{t_2}_{t_1} |\vec{E_1} + \vec{E_2}|^2 (x, t) \: \text{d}t 
    \; , \; C \text{ konstant, } 
\end{equation}

Wird der Wellenansatz \eqref{eqn:ansatz} in \eqref{eqn:props} eingesetzt, so ergibt sich aufintegriert

\begin{equation}
    I_\text{ges} = 2 \cdot C \cdot \vec{E_0}^2 (1 + \text{cos}(\delta_2 - \delta_1))
\end{equation}

mit dem sogenannten Interferenzterm $2 C \vec{E_0}^2$.

Die Gesamtintensität $I_\text{ges}$ weicht also um bis zu $\pm 2 C \vec{E_0}^2$ von ihrem Mittelwert
$2 C \vec{E_0}^2$ ab und verschwindet für den Fall

\begin{equation}
    \delta_2 - \delta_1 = (2n + 1)\pi \: , \; n \in \symbb{N}_0 \; .
\end{equation}

Da für inkohärentes Licht der Interferenzterm bei der zeitlichen Mittelung verschwindet, wird kohärentes
Laserlicht verwendet.

