\section{Auswertung}
\label{sec:Auswertung}

\subsection{Bestimmung der Wellenlänge des Diodenlasers}

Es werden $10$ Messungen des Abstandes $\symup{\Delta}d$ für die Zählung von $z \approx 3000$
Interferenzmaxima durchgeführt. Der Hebelübersetzungsfaktor beträgt dabei $C_\text{H} = 5,046$
und es ergibt sich 

\begin{equation}
    \symup{\Delta} d_\text{H} = \frac{\symup{\Delta}d}{C_\text{H}} \; .
\end{equation}

Die zugehörigen Messwerte sind zusammen mit den nach Gleichung \eqref{eqn:eins} berechneten
Wellenlängen $\lambda$ in Tabelle \ref{tab:mess1} eingetragen.

\begin{table}
    \centering
    \caption{Messwerte zur Wellenlängenbestimmung}
    \label{tab:mess1}
    \sisetup{table-format=2.1}
    \begin{tabular}{c c c c c c}
    \toprule
    $z$ & $d_1 \,/\, \si{\milli\meter}$ & $d_2 \,/\, \si{\milli\meter}$
    & $\symup{\Delta}d \,/\, \si{\milli\meter}$ & $\symup{\Delta}d_\text{H} \,/\, \si{\milli\meter}$
    & $\lambda \,/\, \si{\nano\meter}$\\
    \midrule 
        2999 & 2,00 & 7,02 & 5,02 & 0,99 & 663,45 \\
        3000 & 2,00 & 7,03 & 5,03 & 1,00 & 664,55 \\
        3003 & 8,00 & 2,91 & 5,09 & 1,01 & 671,81 \\
        3000 & 2,50 & 7,54 & 5,04 & 1,00 & 665,87 \\
        3001 & 7,54 & 2,49 & 5,05 & 1,00 & 666,97 \\
        3000 & 2,49 & 7,53 & 5,04 & 1,00 & 665,87 \\
        3001 & 7,53 & 2,49 & 5,04 & 1,00 & 665,65 \\
        3016 & 2,49 & 7,55 & 5,06 & 1,00 & 664,97 \\
        3000 & 7,55 & 2,51 & 5,04 & 1,00 & 665,87 \\
        3000 & 2,51 & 7,55 & 5,04 & 1,00 & 665,87 \\
    \bottomrule
    \end{tabular}
    \end{table}

Der Mittelwert der Wellelnlängen ergibt sich als

\begin{equation*}
    \bar{\lambda} = \SI{666.1 +- 0.7}{\nano\meter} \; .
\end{equation*}

\subsection{Bestimmung des Brechungsindizes}

Die Länge der Messkammer beträgt $b = \SI{50}{\milli\meter}$.
Für eine Druckdifferenz $p - p' = \SI{0.6}{\bar}$ wurden zehn Impulsmaximaanzahlen $z$ gemessen
und sind in Tabelle \ref{tab:mess2} aufgelistet.

\begin{table}
    \centering
    \caption{Messwerte zur Brechungsindexbestimmung}
    \label{tab:mess2}
    \sisetup{table-format=2.1}
    \begin{tabular}{c}
    \toprule
    $z$ \\
    \midrule 
        25\\
        24\\
        24\\
        23\\
        23\\
        24\\
        24\\
        23\\
        21\\
        23\\
    \bottomrule
    \end{tabular}
    \end{table}

Ihr Mittelwert beträgt

\begin{equation*}
    \bar{z} = 23,4 \; .
\end{equation*}

Unter normalen Bedingungen mit

\begin{align*}
    p_0 &= \SI{1.0132}{\bar} \\
    T_0 &= \SI{273.15}{\kelvin} \\
    T &= \SI{298.15}{\kelvin}
\end{align*}

ergeben sich $\symup{\Delta}n$ und $n$ mit den Gleichungen \eqref{eqn:delta} und \eqref{eqn:brech} zu

\begin{align*}
    \symup{\Delta}n &= (\num{0.15587 +- 0.00016}) \cdot 10^{-3} \\
    n &= \num{1.00028729 +- 0.00000030} \; .
\end{align*}





