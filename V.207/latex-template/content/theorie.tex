\section{Theorie}
\label{sec:Theorie}

Bei diesem Versuch soll die Temperaturabhängigkeit der dynamischen 
Viskosität von destilliertem Wasser mit Hilfe der Kugelfall Methode nach 
Höppler ermittelt werden. Dazu wird die Apperaturkonstante des 
verwendeten Viskosimeters ermittelt. Es muss desweiteren mit Hilfe der 
Reynoldzahl überprüft werden, ob laminare Strömungsverhältnisse gegeben 
sind. \\
\\Zunächst werden die Kräfte betrachtet, die auf einem Körper wirken, der
sich durch eine Flüssigkeit bewegt: 

\begin{equation}
F_{ges} = F_g - F_a - F_r
\end{equation}

Dabei bezeichnet $F_g$ die Gravitationskraft, $F_a$ die Auftriebskraft und 
$F_r$ die Reibungskraft, welche hier zunächst ohne Beweis als Stokesche 
Reibung angenommen wird. Somit ergibt sich: 

\begin{equation}
F_{ges} = \rho _K\cdot V_K\cdot g - (\rho _K - \rho _F)\cdot V_K - 6\pi\cdot \eta \cdot v\cdot r
\end{equation}

Dabei ist $\rho _K$ die Dichte der Kugel, $V_K$ das Volumen der Kugel, 
$\rho _F$ die Dichte der verwendeten Flüssigkeit, $\eta$ die Viskosität, 
$v$ die Geschwindigkeit mit der die Kugel fällt und $r$ ihr Radius. 
Die Geschwindigkeit $v$ wird konstant, sobald sich ein Kräftegleichgewicht 
zwischen den wirkenden Kräften einstellt. \\
\\Der Begriff Viskosität $\eta$ lässt sich als "Zähflüssigkeit" interpretieren.
Diese Größe ist materialabhängig, was intuitiv klar ist, wenn man betrachtet, 
dass sich zum Beispiel Honig sehr viel zähflüssiger bewegt als Wasser, was bedeutet, 
dass die Viskosität von Honig weitaus höher ist. 
Demnach werden Bewegungen von Flüssigkeiten durch die dynamische Viskosität dieser 
charakterisiert. 
Bestimmen lässt sich $\eta$ zum Beispiel mit dem Kugelfallviskosimeter nach 
Höppler, welches hier verwendet wird. Dabei gilt die folgende empirische Gleichung: 

\begin{equation}
\eta = K\cdot (\rho _K - \rho _F)\cdot t
\label{eqn:Viskosität}
\end{equation}

Mit der Falldauer $t$, sowie der Apperaturkonstanten $K$.
Die dynamische Viskosität wird als dynamisch bezeichnet, weil sie stark 
temperaturabhängig ist. Häufig gilt die Andradesche Gleichung:

\begin{equation}
\eta (T) = A\cdot exp{\left(\frac{B}{T}\right)}
\end{equation}

Dabei sind A und B Konstanten. 

Diese Zusammenhänge gelten allerdings nur für sogenannte "Newtonsche Fluide",
in denen der Geschwindigkeitsverlauf linear ist. Vorraussetzung für solche 
Fluide sind laminare Strömungsverhältnisse, bei denen die Flüssigkeitsschichten
wirbelfrei aneinander abgleiten. Erst wenn solche Strömungsverhältnisse gegeben 
sind, kann die Reibungskraft als Stokesche Reibung angenommen werden. 
Ob eine Strömung laminar oder turbulent ist, kann anhand der Reynolds-Zahl 
abgeschätzt werden. 

\begin{equation}
Re = \frac{\rho _F\cdot v_m\cdot 2r}{\eta}
\label{eqn:Reynolds}
\end{equation}

Dabei bezeichnet $\rho _F$ die Dichte und $\eta$ die Viskosität der Flüssigkeit. 
$v_m$ ist die mittlere Geschwindigkeit der Kugel und $r$ ihr Radius.
Erhält man für $Re$ einen Wert, der unter 1160 liegt, so kann man davon ausgehen, 
dass die Strömung laminar ist. 

\cite{sample}
