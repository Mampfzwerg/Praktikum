\section{Diskussion}
\label{sec:Diskussion}

Der Literaturwert der Viskosität von Wasser bei 20°C Raumtemperatur
liegt bei etwa \SI{0.001}{\kilo\gram\per\meter\per\second}. 
Der ermittelte Wert von \SI{0.001413+-0.000014}{\kilo\gram\per\meter\per\second}
weicht damit um \SI{29.23}{\percent} von dem Literaturwert ab. 
Diese Abweichung lässt sich auf Ungenauigkeit bei der Zeitmessung und einer
zu großen Fallzeit zurückführen. 
Diese zu große Fallzeit ensteht dadurch, dass die Kugel während der
gesamten Strecke von \SI{10}{\centi\meter} durch den Kontakt mit der Rohrwand 
und durch kleine nicht vollständig entfernte Luftblasen gebremst wird.
Verwirbelungen im System können ausgeschlossen werden, da die Reynoldzahlen 
für alle Temperaturen unterhalb des kritischen Wertes von 1160 liegen. 
Die Reynoldzahl der kleinen Kugel liegt deutlichen über dem Wert 
der großen Kugel. Dies liegt daran, dass die Fallzeit der kleinen
Kugel deutlich kleiner ist, was in einer höheren Geschwindigkeit resultiert.
Der Fehler der Apperaturkonstante ist mit \SI{1.32}{\percent} recht gering.
Mit diesen Messungen konnten die Paramter der Andradeschen Gleichung mit einen 
Fehler von \SI{17.56}{\percent} für A und \SI{3.39}{\percent} für B bestimmt werden
und somit die Temperaturabhänigkeit der Viskosität bestätigt werden. 