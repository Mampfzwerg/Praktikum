\section{Diskussion}
\label{sec:Diskussion}

Bei der Untersuchung der Energieverteilung der beschleunigten Elektronen ist deutlich zu 
erkennen, dass die aufgenommenen Werte stark von den erwarteten Werte abweichen. Gewünscht wäre 
eine nahezu gleiche Energie für alle Elektronen, welche allerdings nicht gegeben ist. Stattdessen
ergab sich eine Verteilung in der Peaks nur schwer zu erkennen sind. 
Die berechnete Anregungsenergie der Quecksilber Atome und der dazugehörige Literaturwert ([2])
lauten 

\begin{align*}
U_1 &= \SI{2.76}{\eV},\\
U_\text{1, Lit} &= \SI{4.9}{\eV}.
\end{align*}

Es ergibt sich eine prozentuale Abweichung von etwa $\SI{43.67}{\percent}$. Daraus lässt sich schließen, 
dass die Bestimmung der Anregungsenergie mit Hilfe des Franck-Hertz-Versuches nicht sehr genau ist. Dies
könnte an einem eventuell falschen Aufbau liegen, wie am Ende der Versuchs festgestellt wurde. Außerdem 
waren sowohl die Brems- als auch die Beschleunigungsspannung nicht optimal abzulesen. Außerdem war es 
recht schwierig die Temperatur im gewünschten Bereich konstant zu halten, was ebenfalls einen negativen 
Einfluss auf die Messung gehabt haben kann. \\
Für die im letzten Versuchsteil zu bestimmende Ionisationsspannung ergeben sich die Werte([3]):

\begin{align*}
U_\text{ion} &= \SI{5.55}{\volt}\\
U_\text{ion,Lit} &= \SI{10.438}{\volt}
\end{align*}

Die prozentuale Abweichung ist auch hier recht hoch mit $\SI{47}{\percent}$. Hier fließt eine weitere 
Unsicherheit durch die recht ungenau Extrapolation mit ein. Bereits für kleine Änderungen in der 
Steigung ergeben sich bereits große Unterschiede für den Schnittpunkt der Geraden mit der x-Achse. \\
Insgesamt lässt sich sagen, dass die Methode des Franck-Hertz-Versuches zur Bestimmung von 
Anregungsenergie und Ionisationsspannung recht ungenau zu sein scheint. Dies liegt an einen 
eventuellen systematischen Fehler und durch mehrere einfließende Unsicherheiten. 