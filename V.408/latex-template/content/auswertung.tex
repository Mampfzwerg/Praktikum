\section{Auswertung}
\label{sec:Auswertung}

\subsection{Bestätigung des Linsengesetzes}

Bei der ersten Messung wurden $g$, $b$ und $B$ bestimmt. Unter Verwendung des Linsengesetzes 
kann zunächst $\frac{1}{f}$ und schließlich $f$ bestimmt werden. Für die Linse bekannter
Brennweite ($\SI{0.150}{\meter}$) ergibt sich somit Tabelle \ref{tab:mess1}. Das Abbildungsgesetz 
kann überprüft werden, indem für jeden Messwert $V_1 = \frac{b}{g}$ und $V_2= \frac{B}{G}$
bestimmt wird. Weiter wird die Differenz $\vert\Delta V \vert$ berechnet. Hierbei erkennt 
man, dass diese Differenz für jeden der Messwerte bei der Linse sehr klein ist.\\

\begin{table}
    \centering
    \caption{Messwerte und Brennweite der Linse mit bekannter Brennweite.}
    \label{tab:mess1}
    \sisetup{table-format=2.1}
    \begin{tabular}{c c c c c c c}
    \toprule
    $ g \;/\; \si{\meter} $ & $b \;/\; \si{\meter}$ &
    $ B \;/\; \si{\meter} $ & $V_1$ & $V_2$ & $\vert \Delta V \vert$ & $f \;/\; \si{meter}$\\
    \midrule 
        0,25 & 0,465 & 0,0450 & 1,860 & 1,500 & 0,360 & 0.163\\
        0,30 & 0,356 & 0,0260 & 1,867 & 0,867 & 0,320 & 0.163\\
        0,35 & 0,300 & 0,0190 & 0,857 & 0,633 & 0,224 & 0.162\\
        0,40 & 0,277 & 0,0160 & 0,693 & 0,533 & 0,159 & 0.164\\
        0,45 & 0,257 & 0,0125 & 0,571 & 0,417 & 0,154 & 0.164\\
        0,50 & 0,243 & 0,0110 & 0,486 & 0,367 & 0,119 & 0.164\\
        0,55 & 0,233 & 0,0095 & 0,424 & 0,317 & 0,107 & 0.164\\
        0,60 & 0,226 & 0,0080 & 0,377 & 0,267 & 0,110 & 0.164\\
        0,65 & 0,218 & 0,0070 & 0,335 & 0,233 & 0,102 & 0.163\\
        0,70 & 0,205 & 0,0065 & 0,293 & 0,217 & 0,076 & 0.159\\          
    \bottomrule
    \end{tabular}
    \end{table}

Aus Tabelle \ref{ta:mess1} folgt somit für den Brennpunkt $f$ als Mittelwert 

\begin{equation*}
\SI{0.163+-0.002}{\meter}.
\end{equation*}

Werden die Wertepaare $(g_i, b_i)$ auf die Achsen eines Koordinatensystems aufgetragen, und 
die zugehören Paare miteinander verbunden, so ergibt sich für die Linse bekannter 
Brennweite Abbildung \ref{fig:plot1}.

\begin{figure}
  \centering
  \includegraphics{content/plot1.pdf}
  \caption{Wertepaare f bekannt.}
  \label{fig:plot1}
\end{figure}

Es ein Ausreißer zu erkennen (cyan Gerade), der beim Ablesen der Brennweite außer Acht gelassen wird. 
Der Schnittpunkt aller Geraden der Messwerte bestimmt den Brennpunkt $f$ der Linse. Dieser
Schnittpunkt wird so genau wie möglich abgelesen. Es ergibt sich der Punkt 

\begin{equation*}
(\SI{0.16}{meter} \vert \SI{0.16}{\meter}).
\end{equation*}

und da die beiden Werte gleich sind, ergibt sich 

\begin{equation*}
f = \SI{0.16}{\meter}.
\end{equation*}

\subsection{Methode von Bessel}

Bei der zweiten Messung wurden die beiden Positionen mit scharfem Bild für ungefiltertes, 
blaues und rotes Licht nach der Methode nach Bessel bestimmt. Aus diesen Messwerten lässt 
sich nach Formel --- die Brennweite der Linse bestimmen. Alles Messwerte, sowie die für die
einzelne Messwerte berechneten Brennweiten sind in Tabelle \ref{tab:mess2}, 
\ref{tab:mess3} und \ref{tab:mess4} dargestellt. 

\begin{table}
    \centering
    \caption{Besselmethode ungefiltertes Licht.}
    \label{tab:mess2}
    \sisetup{table-format=2.1}
    \begin{tabular}{c c c c c c c c c}
    \toprule
    $ e \;/\; \si{\meter} $ & $g_1 \;/\; \si{\meter}$ &
    $ b_1 \;/\; \si{\meter}$ & $g_2 \;/\; \si{\meter}$ & 
    $ b_2 \;/\; \si{\meter}$ & $\vert g_1 - b_1 \vert$ & 
    $ \vert g_2 - b_2 \vert$ & $ f_1 \;/\; \si{\meter}$ &
    $ f_2 \;/\; \si{\meter}$\\
    \midrule 
        0,900 & 0,2120 & 0,6880 & 0,687 & 0,213 & 0,476 & 0,474 & 0,162 & 0,163\\
        0,850 & 0,2170 & 0,6330 & 0,632 & 0,218 & 0,416 & 0,414 & 0,162 & 0,162\\
        0,820 & 0,2240 & 0,5960 & 0,598 & 0,222 & 0,372 & 0,376 & 0,163 & 0,162\\
        0,800 & 0,2270 & 0,5730 & 0,572 & 0,228 & 0,346 & 0,344 & 0,163 & 0,163\\
        0,770 & 0,2310 & 0,5390 & 0,532 & 0,238 & 0,308 & 0,294 & 0,162 & 0,164\\
        0,750 & 0,2365 & 0,5135 & 0,510 & 0,240 & 0,277 & 0,270 & 0,162 & 0,163\\
        0,720 & 0,2470 & 0,4730 & 0,468 & 0,252 & 0,226 & 0,216 & 0,162 & 0,164\\
        0,700 & 0,2550 & 0,4450 & 0,445 & 0,255 & 0,190 & 0,190 & 0,162 & 0,162\\
        0,675 & 0,2720 & 0,4030 & 0,402 & 0,273 & 0,131 & 0,129 & 0,162 & 0,163\\
        0,650 & 0,3070 & 0,3430 & 0,352 & 0,298 & 0,036 & 0,054 & 0,162 & 0,161\\          
    \bottomrule
    \end{tabular}
    \end{table}

\begin{table}
    \centering
    \caption{Besselmethode rotes Licht.}
    \label{tab:mess3}
    \sisetup{table-format=2.1}
    \begin{tabular}{c c c c c c c c c}
    \toprule
    $ e \;/\; \si{\meter} $ & $g_1 \;/\; \si{\meter}$ &
    $ b_1 \;/\; \si{\meter}$ & $g_2 \;/\; \si{\meter}$ & 
    $ b_2 \;/\; \si{\meter}$ & $\vert g_1 - b_1 \vert$ & 
    $ \vert g_2 - b_2 \vert$ & $ f_1 \;/\; \si{\meter}$ &
    $ f_2 \;/\; \si{\meter}$\\
    \midrule 
        0,90 & 0,214 & 0,686 & 0,683 & 0,217 & 0,472 & 0,466 & 0,163 & 0,165\\
        0,85 & 0,220 & 0,630 & 0,627 & 0,223 & 0,410 & 0,404 & 0,163 & 0,164\\
        0,80 & 0,228 & 0,572 & 0,570 & 0,230 & 0,344 & 0,340 & 0,163 & 0,164\\
        0,75 & 0,240 & 0,510 & 0,509 & 0,241 & 0,270 & 0,268 & 0,163 & 0,164\\
        0,70 & 0,261 & 0,439 & 0,442 & 0,258 & 0,178 & 0,184 & 0,164 & 0,163\\         
    \bottomrule
    \end{tabular}
    \end{table}

\begin{table}
    \centering
    \caption{Besselmethode blaues Licht.}
    \label{tab:mess4}
    \sisetup{table-format=2.1}
    \begin{tabular}{c c c c c c c c c}
    \toprule
    $ e \;/\; \si{\meter} $ & $g_1 \;/\; \si{\meter}$ &
    $ b_1 \;/\; \si{\meter}$ & $g_2 \;/\; \si{\meter}$ & 
    $ b_2 \;/\; \si{\meter}$ & $\vert g_1 - b_1 \vert$ & 
    $ \vert g_2 - b_2 \vert$ & $ f_1 \;/\; \si{\meter}$ &
    $ f_2 \;/\; \si{\meter}$\\
    \midrule 
        0,90 & 0,212 & 0,688 & 0,687 & 0,213 & 0,476 & 0,474 & 0,162 & 0,163\\
        0,85 & 0,218 & 0,632 & 0,631 & 0,219 & 0,414 & 0,412 & 0,162 & 0,163\\
        0,80 & 0,227 & 0,573 & 0,572 & 0,226 & 0,346 & 0,346 & 0,163 & 0,163\\
        0,75 & 0,236 & 0,514 & 0,511 & 0,239 & 0,278 & 0,272 & 0,162 & 0,163\\
        0,70 & 0,250 & 0,445 & 0,442 & 0,258 & 0,195 & 0,184 & 0,161 & 0,163\\         
    \bottomrule
    \end{tabular}
\end{table}

Aus diesen Tabellen ergeben sich als Mittelwerte für die Brennweiten

\begin{align*}
f &= \SI{0.1621+-0.0003}{\meter},\\
f_\text{rot} &= \SI{0.1639+-0.0006}{\meter},\\
f_\text{blau} & = \SI{0.1627+-0.0001}{\meter}.
\end{align*}

\subsection{Methode von Abbe}
Bei der dritten Messung soll die Brennweite, sowie die Positionen der Hauptebene eines Linsensystems 
bestimmt werden. Aus den gemessen Werten für $B$ lässt sich der Abbildungsmaßstab $V$ bestimmen. Daraus 
wiederum lassen sich die Größen $(1+\frac{1}{V})$ und $(1+V)$ berechnen. Diese Größen sind 
in Tabelle \ref{tab:mess5} eingetragen.

\begin{table}
    \centering
    \caption{Messwerte Methode nach Abbe}
    \label{tab:mess5}
    \sisetup{table-format=2.1}
    \begin{tabular}{c c c c c c}
    \toprule
    $ g' \;/\; \si{\meter} $ & $b' \;/\; \si{\meter}$ &
    $ B \;/\; \si{\meter} $ & $V$ & $1+\frac{1}{V}$ & $1+V$\\
    \midrule 
        0,20 & 0,740 & 0,0570 & 1,90 & 1,526 & 2,90\\
        0,25 & 0,555 & 0,0340 & 1,13 & 1,882 & 2,13\\
        0,30 & 0,470 & 0,0230 & 0,77 & 2,304 & 1,77\\
        0,35 & 0,433 & 0,0180 & 0,60 & 2,667 & 1,60\\
        0,40 & 0,395 & 0,0190 & 0,63 & 2,579 & 1,63\\
        0,45 & 0,385 & 0,0120 & 0,40 & 3,500 & 1,40\\
        0,47 & 0,372 & 0,0125 & 0,42 & 3,400 & 1,42\\
        0,50 & 0,370 & 0,0100 & 0,33 & 4,000 & 1,33\\
        0,55 & 0,353 & 0,0090 & 0,30 & 4,333 & 1,30\\
        0,60 & 0,350 & 0,0080 & 0,27 & 4,750 & 1,37\\          
    \bottomrule
    \end{tabular}
    \end{table}

Weiter wird $g'$ gegen $(1+1/V)$ und $b'$ gegen $(1+V)$ aufgetragen, woraus Abbildung \ref{fig:plot2}
und \ref{fig:plot3} resultieren.  

\begin{figure}
  \centering
  \includegraphics{content/plot2.pdf}
  \caption{Messwerte $g'$ gegen $1+1/V)$.}
  \label{fig:plot2}
\end{figure}

\begin{figure}
  \centering
  \includegraphics{content/plot3.pdf}
  \caption{Messwerte $b'$ gegen $1+V$.}
  \label{fig:plot3}
\end{figure}

Es wurde jeweils eine lineare Regression mittels Python durchgeführt. Die dabei erhaltenen Parameter 
geben hierbei nach Formel --- und --- direkt den Brennpunkt und die Lage der Hauptebenen an. Es ergibt sich
für 

\begin{align*}
g' &= g + h = f_1 \cdot \left(1+\frac{1}{V}\right)+h,\\
b' &= b + h = f_2 \cdot \left(1+V\right)+h'  
\end{align*}

und somit 

\begin{align*}
f_1 &= \SI{0.120+-0.009}{\meter}, \\
f_2 &= \SI{0.240+-0.011}{\meter}, \\
h & = \SI{0.035+-0.028}{\meter},\\
h'&= \SI{0.040+-0.020}{\meter}.
\end{align*}

Es ist erkennbar, dass $f_1$ und $f_2$ nicht denselben Wert besitzen, sodass für die letztliche 
Bestimmung der Brennweite der Mittelwert gebildet wird. Es folgt somit

\begin{equation*}
f = \SI{0.180+-0.060}{\meter}.
\end{equation*}

Die Position der Hauptebenen $h$ und $h'$ beziehen sich auf den Referenzpunkt $A$. Dieser hat gerade die 
Mittelebene der dünnen Sammellinse. 