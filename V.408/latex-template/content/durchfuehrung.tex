\section{Durchführung}
\label{sec:Durchführung}

Der Versuchsaufbau besteht im Wesentlichen aus dem Gegenstand, einer optischen L-Schablone,
die von einer Halogenleuchte beleuchtet wird und dem Schirm, der wie der Gegenstand auf
einer optischen Bank verschiebbar aufgehängt ist. \\

Zunächst werden zur Verfizierung des Abbildungsgesetzes und der Linsengleichung
mit einer Sammellinse bekannter Brennweite $f$ für verschiedene Gegenstandsweiten $g$
der Schirm jeweils so justiert, dass das Bild scharf ist und die Bildweiten $b$ und Bildgrößen $B$ 
gemessen.\\

Weiterhin wird die Bessel-Methode zur Brennweitenbestimmung durchgeführt.
Dabei wird für verschiedene feste Abstände $e$ zwischen Gegenstand und Schirm eine Sammellinse so verschoben,
dass auf dem Schirm ein scharfes Bild abbgebildet wird. Für die jeweils zwei Positionen werden
jeweils die Gegenstandsweiten $g_1$ und $g_2$, sowie die Bildweiten $b_1$ und $b_2$ gemessen.
Dies wird zudem nochmals jeweils mit blau und rot gefiltertem Licht durchgeführt.
Außerdem ist zu beachten, dass $e$ mindestens vier mal größer als die Brennweite $f$ der Linse sein muss.\\

Zuletzt wird die Abbe-Methode angewandt. Dazu wird zwischen den Gegenstand und eine Sammellinse mit
Brennweite $f = 100$ eine Zerstreulinse mit Brennweite $f = -100$ geklemmt, die einen konstanten
Abstand zur Sammellinse behalten soll. Es werden ein Referrenzpunkt $A$ für die Messung festgelegt
und für verschiedene Gegenstandsweiten $g'$ der Schirm so justiert, dass ein scharfes Bild entsteht und die
jeweils zugehörige Bildweite $b'$ und Bildgröße $B$ gemessen. 


