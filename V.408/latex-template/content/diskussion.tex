\section{Diskussion}
\label{sec:Diskussion}

In der ersten Messung sollte die bekannte Brennweite von $f = \SI{0.150}{\meter}$ verifiziert und das 
Abbildungsgesetz bestätigt werden. Aufgrund der sehr kleinen Differenz $\Delta V$ kann das Abbildungsgesetz
bestätigt werden. Bei der ersten Messung wurde eine Brennweite von 

\begin{equation*}
f_\text{exp} = \SI{0.163+-0.002}{\meter}
\end{equation*}

festgestellt. Dies weicht nur um $\SI{8.67}{\percent}$ von der tatsächlichen Brennweite von 
$f_\text{theo}=\SI{0.150}{\meter}$ ab. Beim Bestimmen der Brennweite als Schnittpunkt der 
Gerade wurde diese zu 

\begin{equation*}
f_\text{Sch} = \SI{0.16}{\meter} 
\end{equation*}

bestimmt. Die Werte der Brennweiten unterscheiden sich bei den beiden Verfahren lediglich um 
$\SI{1.87}{\percent}$ bzw. um $\SI{1.84}{\percent}$. 

Bei der Verifzierung der Brennweite mittels der Methode nach Bessel, wurde diese zu 

\begin{equation*}
f = \SI{0.1621+-0.0003}{\meter}
\end{equation*}

bestimmt. Diese weicht um $\SI{8.06}{\percent}$ von der tatsächlichen Brennweite ab. Generell 
können all diese Abweichungen durch ein möglicherweise falsch angesetztes Augenmaß zur Bestimmung, ob das 
Bild scharf ist, erklärt werden. Weiter wurden die Brennweite der Linse für rotes und blaues Licht 
bestimmt. Es ergaben sich die Brennweiten

\begin{align*}
f_\text{rot} &= \SI{0.1639+-0.0006}{\meter},\\
f_\text{blau} & = \SI{0.1627+-0.0001}{\meter}.
\end{align*}

Diese liegen sehr nah zusammen und weichen fast gar nicht voneinander ab. 

Bei der Bestimmung der Brennweite eines Linsensystems nach der Methode nach Abbe ergeben sich 
zwei Werte $f_1$ und $f_2$ für die Brennweiten, aus denen ein Mittelwert gebildet wurde:

\begin{equation*}
f = \SI{0.180+-0.060}{\meter}.
\end{equation*}

Dieses Ergebnis liegt im erwarteten Bereich von ungefähr $\SI{0.200}{\meter}$, da ein 
sehr kleiner Abstand zwischen den beiden Linsen gewählt wurde und die Brechkräfte sich dann
aufaddieren. 