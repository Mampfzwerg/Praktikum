\section{Diskussion}
\label{sec:Diskussion}

Mögliche Fehlerquellen des Versuches sind die Ableseungenauigkeiten des
einzustellenden Abstandes $x_0$ und Druckes $p$.

Die durch die mittleren Reichweiten

\begin{align*}
    R_{m,1} &= \SI{1.86 +- 0.15}{\centi\meter} \\
    R_{m,2} &= \SI{1.95 +- 0.21}{\centi\meter}
\end{align*}

bestimmten Energiewerte

\begin{align*}
    E_{\alpha, 1} &= \SI{3.30 +- 0.18}{\mega\eV} \\
    E_{\alpha, 2} &= \SI{3.41 +- 0.24}{\mega\eV}
\end{align*}

weichen von den durch den Energieverlust bestimmten Energiewerten

\begin{align*}
    E_{\alpha, R_{\alpha, 1}} &= \SI{2.47 +- 0.20}{\mega\eV} \\
    E_{\alpha, R_{\alpha, 2}} &= \SI{2.27 +- 0.22}{\mega\eV}
\end{align*}

jeweils um $\SI{33.6}{\percent}$ und $\SI{50.2}{\percent}$ ab.

Beide  in gleicher Weise bestimmten Energiewertepaare verschiedener Messreihen liegen näher beieinander, als 
zu dem jeweils anders bestimmten Werten innerhalb der eigenen Messreihe.
Zu erwarten wäre hingegen das genaue Gegenteil. Dies kann daran liegen, dass die Beziehung \eqref{eqn:energie},
die nur für Energien $\leq \SI{2.5}{\mega\eV}$ gilt, für die gemessenen Reichweiten gar nicht angenommen werden
darf.  

Zur Statistik des radioaktiven Zerfalls lässt sich sagen, dass beide Verteilungen recht nah an den Theoriewerten
liegen. Die Abweichungen scheinen bei der Poissonverteilung jedoch kleiner zu sein.

