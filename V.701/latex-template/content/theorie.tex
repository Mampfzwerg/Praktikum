\section{Theorie}
\label{sec:Theorie}

Ziel dieses Versuchs ist die Bestimmung der Energie von $\alpha$-Strahlung über die Messung von deren 
Reichweite in Luft. \\
Ein $\alpha$-Teilchen verliert über zwei Aspekte seine Energie.
Zunächst über elastische Stöße mit dem Medium, welches es durchläuft. Dieser Prozess, der als Rutherford-Streuung
bekannt ist, spielt für den Energieverlust eine nebensächlichen Rolle, da die Größe der Atomkerndichte im Material 
sehr gering ist und somit ein Zusammenstoß relativ unwahrscheinlich ist.\\
Den zweiten, weitaus wichtigeren, Aspekt stellen Ionisationsprozesse, sowie Anregung uns Dissoziation von Molekülen
im Material dar. Dabei hängt auch hier der Energieverlust pro Wegstück von der Dichte des Materials und der 
Energie $E_\alpha$ der $\alpha$-Strahlung ab. Bei einer hohen Geschwindigkeit und somit einer großen 
Energie der Strahlung, ist es jedoch wahrscheinlicher, dass es zu keiner Wechselwirkung kommt, da die Zeit, 
in der sich der Heliumkern in Wechselwirkungsnähe befindet, geringer ist.\\
Bei hinreichend großen Energien kann der Energieverlust pro Wegstück über die Bethe-Block Gleichung

\begin{equation}
-\frac{\symup{d}E_\alpha}{\symup{d}x} = \frac{z²e⁴}{4\pi\epsilon_0 m_\text{e}} \frac{nZ}{v²}\ln{\left(\frac{2m_\text{e}v²}{I}\right)}
\label{eqn:Bethebloch}
\end{equation}

bestimmt werden, wobei $z$ die Ladung, $v$ die Geschwindigkeit, $Z$ die Ordnungszahl, $n$ die Teilchendichte 
und $I$ die Ionisationsenergie des Targetgases ist. Quantenmechanische Prozesse wie zum Beispiel der Kernspin 
werden ignoriert. \\
Die Reichweite der $\alpha$-Strahlung ergibt sich dann über das Integral 

\begin{equation}
R = \int_{0}^{E_\alpha} -\frac{\symup{d}E_\alpha}{\symup{d}x} \symup{d}E_\alpha.
\label{eqn:Reichweite}
\end{equation}

Da bei geringeren Energien vermehrt Ladungsaustauschprozesse stattfinden, verliert die Bethe-Bloch-Gleichung 
ihre Gültigkeit und somit wird die mittlere Reichweite über empirisch gewonnene Kurven bestimmt. Für sie gilt 
für $\alpha$-Stahlung in Luft die Gleichung 

\begin{equation}
R_\text{m}= \num{3.1}\cdot E_\alpha^\frac{3}{2},
\end{equation}

wobei die Energie in Megaelektronenvolt angegeben in einem Bereich unter $\SI{2.5}{\mega\eV}$ liegen sollte und 
$R_\text{m}$ in Millimeter angegeben wird. Unter konstanter Temperatur und konstantem Volumen ist die Reichweite 
propotional zum umgebenden Druck $p$. Folglich kann zur Ermittlung der Reichweite eine Absorptionsmessung 
gemacht werden, bei der der Druck variiert wird. Dadurch gilt für einen festen Abstand $x_0$ zwischen Detektor 
und Strahler die effektive Weglänge 

\begin{equation}
x_\text{eff} = x_0 \frac{p}{p_0},
\label{eqn:Weglaenge}
\end{equation}

welche durch den Normaldruck $p_0 = \SI{1013}{\milli\bar}$ [1] beschrieben wird. 