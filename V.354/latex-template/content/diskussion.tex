\section{Diskussion}
\label{sec:Diskussion}

Weitesgehend zeigt sich in allen Versuchsteilen eine hohe Übereinstimmung 
zwischen den gemessenen und theoretisch berechneten Werten.

Der experimentell ermittelte Wert des Dämpfungswiderstandes $R_\text{eff}$ kann
nicht mit einem theoretischen Wert verglichen werden, da der Innenwiderstand 
des Frequenzgenators berücksichtigt werden muss. Dadurch addiert sich zum Widerstand
$R_1 = \SI{48.1}{\ohm}$ ein unbekannter Wert. Da der ermittelte Wert für $R_\text{eff}$
größer als $R_1$ ist, lässt sich allerdings zumindest feststellen, dass der ermittelte
Wert für den Dämpfungswiderstandes nicht falsch ist. \\
Die Abweichung des Dämpfungswiderstandes bei dem aperiodischen Grenzfalles zum 
theoretischen Wert beträgt $\SI{19.82}{\percent}$. Dies kann damit erklärt werden, 
dass der aperiodische Grenzfall nicht ganz genau zu bestimmen war. Die Methode 
des Bestimmens setzte ein gewisses Augenmaß voraus, wodurch sich Ungenauigkeiten 
ergeben. Außerdem ist das Potentiometer als regelbarer Widerstand nicht ganz 
genau. \\
Der theoretische Wert der Resonanzüberhöhung ist $\SI{12.74}{\percent}$ größer als der 
experimentell ermittelte Wert. \\
Der theoretische Wert der Halbwertsbreite ist um $\SI{24.67}{\percent}$ kleiner als der 
experimentell ermittelte. Auch bei dieser Rechnung müsste der Innenwiderstand
beachtet werden.

Außerdem wurde bemerkt, dass die Messwerte der Anregespannung $U$ das Verhältnis $\frac{U_C}{U}$
dahingehend beeinflussen, dass keine aussagekräftige Resonanzkurve zustande kommt. Da dies
für die Frequenzen der Fall ist, für die $U$ von dem Wert $\SI{1}{\volt}$ abweicht, wird
dieser Wert für $U$ angenommen. Ursache für diese Fehlerquelle mag die Ungenauigkeit des Tastkopfes
sein. \\
Weiterhin weichen die experimentell ermittelten Frequenzen $f_1$ um $\SI{2.53}{\percent}$,
$f_2$ um $\SI{2.06}{\percent}$ und die Resonanzfrequenz $f_\text{res}$ nur um $\SI{0.26}{\percent}$
von ihren theoretischen Werten nach unten ab. Diese Messung ist also recht präzise.
