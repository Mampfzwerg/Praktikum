\section{Auswertung}
\label{sec:Auswertung}

\subsection{Zeitabhängigkeit der Amplitude}

Die gemessenen Maxima bei einer gedämpften Schwingung sind 
in Tabelle \ref{tab:Messdaten1} zu sehen. 

\begin{table}
\centering
\caption{Messdaten der Maxima der Amplitude}
\label{tab:Messdaten1}
\sisetup{table-format=2.1}
\begin{tabular}{c c}
\toprule
$U \,/\, \si{\volt}$ & $s \,/\, \si{\micro\second}$\\
\midrule
1,74 &   0,0\\
1,44 &  29,6\\
1,20 &  58,8\\
1,10 &  88,2\\
0,84 & 117,2\\
0,68 & 147,2\\
0,56 & 177,2\\
0,46 & 206,2\\
0,38 & 235,2\\
0,30 & 265,2\\
0,22 & 294,2\\ 
0,16 & 324,2\\
0,14 & 353,2\\
0,10 & 382,2\\
0,06 & 412,2\\
0,02 & 441,2\\
0,00 & 471,2\\
\bottomrule
\end{tabular}
\end{table} 

Die Ausgleichsrechnung wird mit der Funktion 

\begin{equation*}
A = A_0 \cdot e^{-2\pi\mu t}
\end{equation*}

durchgeführt. Das Ergebnis ist in Abbildung \ref{fig:gedämpft} zu sehen. 

\begin{figure}
  \centering
  \includegraphics[scale=0.8]{content/plot1.pdf}
  \caption{Exponentielle Regression der Amplitude}
  \label{fig:gedämpft}
\end{figure}

Mittels python ergeben sich die Regressionsparamter zu: 

\begin{align*}
A_0 &= \SI{1.785+-0.036}{\volt},\\
\mu &= \SI{1068.421+-34.320}{\per\second}.
\end{align*}

Mit Formel... lässt sich nun der effiktive Widerstand berechnen.

\begin{equation*}
R_\text{eff} = 4\pi L\mu = \SI{136+-4}{\ohm}
\end{equation*}

Der Fehler ergibt sich dabei durch die Gaußsche Fehlerfortpflanzung zu: 

\begin{equation*}
\symup{\Delta} R_\text{eff} = \sqrt{\left(\frac{\symup{d}R_\text{eff}}{\symup{d}L}\right)²\cdot (\symup{\Delta}L)² +
\left(\frac{\symup{d}R_\text{eff}}{\symup{d}\mu}\right)²\cdot (\symup{\Delta}\mu)²}.
\end{equation*}

Weiterhin wird die Abklingdauer mit Formel... berechnet und 
es ergibt sich:

\begin{equation*}
T_\text{ex} = \frac{1}{2\pi\mu} = \SI{0.149+-0.005e-3}{\second}.
\end{equation*}

Der Fehler ergibt sich hierbei zu: 

\begin{equation*}
\symup{\Delta} T_\text{ex} = \sqrt{\left(\frac{\symup{d}T_\text{ex}}{\symup{d}\mu}\right)²\cdot (\symup{\Delta}\mu)²}.
\end{equation*}

\subsection{Bestimmung des Dämpfungswiderstandes}

Hier wurde der aperiodische Grenzfall untersucht. Dabei wurde der
Dämpfungswiderstand zu 

\begin{equation*}
R_\text{ap} = \SI{3520+-50}{\ohm}
\end{equation*}

bestimmt.
Der theoretische Wert von $R_\text{ap}$ kann mit Formel \eqref{eqn:apth} bestimmt 
werden: 

\begin{equation*}
R_\text{ap,theo} = \SI{4390+-9}{\ohm}.
\end{equation*}

Der Fehler berechnet sich über die Gaußsche Fehlerfortpflanzung: 

\begin{equation*}
\symup{\Delta} R_\text{ap,theo} = \sqrt{\left(\frac{\symup{d}R_\text{ap}}{\symup{d}L}\right)²\cdot (\symup{\Delta}L)² +
\left(\frac{\symup{d}R_\text{ap}}{\symup{d}C}\right)²\cdot (\symup{\Delta}C)²}.
\end{equation*}

\subsection{Frequenzabhängigkeit der Kondensatorspannung}