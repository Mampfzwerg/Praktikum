\section{Theorie}
\label{sec:Theorie}

Ziel des Versuches ist es, gedämpfte und erzwungene Schwingungen eines
RLC-Schwingkreises zu untersuchen und den Dämpfungswiderstand und die 
Frequenzabhängigkeit der Phasenverschiebung zwischen Kondensator- und 
Generatorspannung zu bestimmen. 
Ein solcher Schwingkreis zeichnet sich durch eine Kapazität $C$, 
eine Induktivität $L$ und einen Widerstand $R$ aus. Bei einer 
Schwingung wird beständig Energie zwischen dem B-Feld der Spule und
dem E-Feld des Kondensators ausgetauscht. Der Widerstand $R$ sorgt
dafür, dass die Energie des Systems fortlaufend abnimmt und so die 
Amplituden immer kleiner werden, bis sie irgendwann gänzlich abgeklungen 
sind. 
Mit Hilfe des zweiten Kirchhoffschen Gesetztes kann man eine Gleichung 
für den Schaltkreis aufstellen: 

\begin{equation*}
U_R(t) + U_C(t) + U_L(t) = 0 
\end{equation*}

Durch Umformungen kann eine DGL für das System aufgestellt werden: 

\begin{equation}
L\frac{\symup{d}²I}{\symup{d}t²}+\frac{R}{L} \frac{\symup{d}I}{\symup{d}t}+\frac{I}{LC} = 0
\end{equation}

Eine Lösung der DGL ist: 

\begin{equation}
I(t) = e^{-t\frac{R}{2L}}\left(I_1 e^{it\sqrt{\frac{1}{LC}-\frac{R²}{4L²}}}+I_2 e^{-it\sqrt{\frac{1}{LC}-\frac{R²}{4L²}}} \right).
\label{eqn:Loesung}
\end{equation}

Es gibt drei verschiedene Fälle, die betrachtet werden müssen. 
Der erste Fall ist, dass $\frac{1}{LC} > \frac{R²}{4L²}$. Die Wurzel im Exponenten
bleibt reell und \eqref{eqn:Loesung} geht über in: 

\begin{equation*}
I(t) = A_0 e^{-\frac{R}{2L}t}\cdot \cos\left({2\pi ft+\eta}\right).
\end{equation*}

Hierbei ist $\eta$ eine Phasenverschiebung. Dieser Fall wird auch als 
Schwingfall bezeichnet. 
Wenn nun gilt

\begin{equation*}
\frac{1}{LC} = \frac{R²}{4L²}
\end{equation*}

\begin{equation}
\leftrightarrow R = 2\sqrt{\frac{L}{C}},
\label{eqn:apth}
\end{equation}

spricht man vom aperiodischen Grenzfall. Bei diesem Fall geht $I(t)$ ohne 
Überschwingung am schnellsten gegen null. 

Wird der RLC-Kreis nun von außen durch eine Wechselspannung zum Schwingen 
angeregt, so spricht man von einer erzwungenen Schwingung. In diesem Fall
liegt eine Inhomogenität in der DGL vor. Durch Lösen der resultierenden 
DGL erhält man schließlich: 

\begin{equation}
U_C(\omega) = \frac{U_0}{\sqrt{(1-LC\omega²)²+\omega²R²C²}}
\end{equation}

Diese Lösung hat ein Maximum bei $f = f_\text{res}$, das Resonanzüberhöhung
genannt wird. Die Breite $\symup{\Delta}f$ kann mit

\begin{equation}
\symup{\Delta} f = |f_1 - f_2|
\end{equation}

bestimmt werden. Hierbei gilt: 

\begin{equation}
f_\text{res} = \frac{1}{2\pi}\sqrt{\frac{1}{LC}-\frac{R²}{2L²}}
\end{equation}

und 

\begin{equation}
f_\text{1,2} = \pm \frac{1}{2\pi} \frac{R}{2L}+\sqrt{\frac{R²}{4L²}+\frac{1}{LC}}.
\end{equation}

Außerdem weist die Kondensatorspannung $U_C(t)$ eine Phasenverschiebung gegenüber
der Erregerspannung auf. Diese kann mit 

\begin{equation}
\tan({\phi(\omega)})=\frac{-\omega RC}{1-LC\omega²}
\end{equation}

bestimmt werden. 