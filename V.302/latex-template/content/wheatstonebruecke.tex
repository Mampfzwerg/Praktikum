\section{Wheatstone Brücke}
\label{sec:Wheatstone Brücke}

\subsection{Durchführung}
\label{subsec:Durchführung}

In Abbildung 4 ist der Aufbau einer Wheatstone Brücke schematisch dargestellt.

\begin{figure}
    \centering
    \begin{tikzpicture}[circuit ee IEC, font=\sffamily]
    \draw (0,0) to [battery={info'={$U_s$}}] (0,-6);
    \draw (0,0) -- (3,0);
    \draw (3,0) -- (3, -1);
    \node[contact] at (3,-1) {};
    \draw (1.5, -1) -- (4.5, -1);
    \draw (1.5, -1) to [resistor={info={$R_x$}}] (1.5, -3);
    \draw (1.5, -3) to [resistor={info={$R_2$}}] (1.5, -5);
    \draw (4.5, -1) to [resistor={info={$R_3$ \\ $R_4$}}] (4.5, -5);
    \node[scale=0.8] at (3,-2.75) {Nullindikator};
    \node[scale=0.8] at (3,-3.2) {$U_b$};
    \node[contact] at (1.5,-3) {};
    \node[contact] at (2.5, -3) {};
    \node[contact] at (3.5,-3) {};
    \draw (1.5, -3) -- (2.5, -3);
    \draw (3.5, -3) -- (4.4, -3);
    \draw (1.5, -5) -- (4.5, -5);
    \node[contact] at (3,-5) {};
    \draw (0, -6) -- (3,-6);
    \draw (3, -6) -- (3,-5);
    \end{tikzpicture}
    \caption{Wheatstone Brücke}
    \label{fig:Wheatstone Brücke}
\end{figure}

Sie besteht demnach aus zwei jeweils parallel geschalteten Widerständen
$R_x$ und $R_3$, und $R_2$ und $R_4$.
Dabei wird die Spannung jeweils nach dem ersten Widerstand gemessen.
Da hier Wechselstrom angelegt ist, werden zwei Widerstände $R_3$ und $R_4$
solange variiert, bis die Brückenspannung $U_b$ ihr Minimum (optimal wäre 
null) erreicht hat. Bei diesem Versuch
werden zwei unbekannte Widerstände ermittelt. Zur Fehlerbestimmung wird 
$R_2$ variiert.

\subsection{Auswertung}
\label{subsec:Auswertung}