\section{Diskussion}
\label{sec:Diskussion}

Die Untersuchung des Selektivverstäkers verlief wie erwartet 
und ohne Auffälligkeiten. \\
Zunächst betrachten wir die Abweichungen der experimentell 
bestimmten Suszeptibilitäten von den theoretischen. Diese finden 
sich in Tabell \ref{tab:Messdaten8}.

\begin{table}
\centering
\caption{Abweichungen zum theoretischem Wert.}
\label{tab:Messdaten8}
\sisetup{table-format=2.1}
\begin{tabular}{c c c}
\toprule
Stoff & $\frac{\chi_\text{theo}-\chi_\text{R}}{\chi_\text{theo}}$ & $\frac{\chi_\text{theo}-\chi_\text{U}}{\chi_\text{theo}}$\\ 
\midrule
$\symup{Dy_2O_3}$       & $\SI{91}{\percent}$ & $\SI{77}{\percent}$\\
$\symup{Nd_2O_3}$       & $\SI{6}{\percent}$ & $\SI{85}{\percent}$\\
$\symup{Gd_2O_3}$       & $\SI{75}{\percent}$ & $\SI{5842}{\percent}$\\
$\symup{C_6O_{12}Pr_2}$ & $\SI{310}{\percent}$ & $\SI{141383}{\percent}$\\
\bottomrule
\end{tabular}
\end{table}

Diese ergeben sich nach 

\begin{equation*}
\nu = \left|\frac{\chi_\text{theo}-\chi_\text{exp}}{\chi_\text{theo}}\right| \cdot 100.
\end{equation*}

Aus Tabelle \ref{tab:Messdaten8} können sehr hohe Abweichungen 
entnommen werden. Dies hat vierlerlei mögliche Gründe. 
Zum einen konnte die Brückenspannung nicht genau auf 0 genährt 
werden, als sie abgeglichen wurde. Außerdem war die 
Messaperatur beinahe willkürlich. Bei kleinen Berührungen veränderte 
sich die gemessene Spannung signifikant, sodass es kaum möglich war
zu bewerten, ob das aktuelle Signal stabil war. Außerdem waren die 
Ergebnisse bei Wiederholung auch nicht rekonstruierbar. Des Weiteren
hat der Verstärker einen Innenwiderstand, der nicht vernachlässigbar ist. 
Ein weiterer Punkt ist, dass die Suszeptibilitäten der verwendeten Proben
temperaturabhängig sind, weswegen diese nicht lange in der Hand 
gehalten werden dürfen, da sonst ein systematischer Fehler auftritt. 
Insgesamt sind die gemessenen Werte also sehr fehleranfällig und 
nicht aussagekräftig. 

