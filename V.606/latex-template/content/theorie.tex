\section{Theorie}
\label{sec:Theorie}

In diesem Versuch werden die Suszeptibilitäten paramagnetischer Substanzen 
mit Hilfe einer Brückenschaltung bestimmt. Außerdem wird die Filterkurve 
des dabei verwendeten Selektivverstärkers untersucht. 

\subsection{Die Suszeptibilität}

Die magnetische Suszeptibilität $\chi$ ist eine dimensionslose Größe, die 
angibt, wie gut ein Material in einem externen Magnetfeld magnetisierbar ist,
d.h. wie sich die Magnetisierung des Materials durch ein externes Magnetfeld 
ändert. Diese Größe ist im Allgemeinem von vielen Variablen abhängig (z.B. der
magnetischen Feldstärke $\vec{H}$ und der Temperatur T) und tensoriell. 

