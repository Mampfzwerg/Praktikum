\section{Durchführung}
\label{sec:Durchführung}

Zur Messung der Suszeptibilität wird die Brückenschaltung
gemäß Abbildung ... verwendet. Die durch diese Schaltung
aufkommenden Störspannungen werden durch einen Frequenzfilter,
einen Bandpass herausgefiltert und die gesuchte
Brückenspannung verstärkt. Zunächst gilt es,
die Durchlassfrequenz des Filters durch Messung der
Ausgangsspannung mit einem Mikrovoltmeter zu bestimmen.
Gemessen wird die Ausgangsspannung $U_A$ in dreißig Schritten
für Frequenzen zwischen Frequenzen von $\nu = \SI{20}{\kilo\hertz}$ und
$\SI{40}{\kilo\hertz}$. Dabei sollten Schritte zwischen den Messungen
nahe des Maximums kleingehalten werden.


