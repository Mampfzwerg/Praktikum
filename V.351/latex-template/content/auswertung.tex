\section{Auswertung}
\label{sec:Auswertung}

\subsection{Die Fourieranalyse}

Im ersten Versuchsteil soll der Abfall der Fourierkoeffizienten $a_k$ und $b_k$
periodischer Sägezahn, Rechtecks- und Dreiecksspannungen untersucht werden.
Dazu werden die Ordnung $k$ und die Spannungsamplitude $U$ gegeneinander
logarithmiert aufgetragen und eine lineare Regression der Form

\begin{equation}
    \text{ln} \left(U \cdot \frac{1}{\si{\milli\volt}} \right) = a \cdot \text{ln}(k) + b
    %hier andere Beschriftung für a und b überlegen
\end{equation}

Dabei wird die Spannung $U$ durch ihre Einheit geteilt, da der Logarithmus nur für
für dimensionslose Größen definiert ist. Zudem ergibt sich umgeformt:

\begin{equation}
    U = \text{e}^b \cdot k^a \cdot \si{\milli\volt}
\end{equation}

Da nun $U$ proportional zu den Fourierkoeffizienten ist, lässt sich die
Stärke deren Abfalls ermitteln.
Die Messdaten der Sägezahnspannung sind dazu in Tabelle... aufgeführt und in 
Abbildung ... gegeneinander aufgetragen.




