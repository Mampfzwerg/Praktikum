\section{Auswertung}
\label{sec:Auswertung}

\subsection{Fourieranalyse}

Im ersten Versuchsteil soll der Abfall der Fourierkoeffizienten $a_k$ und $b_k$
periodischer Sägezahn, Rechtecks- und Dreiecksspannungen untersucht werden.
Dazu werden die Ordnung $k$ und die Spannungsamplitude $U$ gegeneinander
logarithmiert aufgetragen und eine lineare Regression der Form

\begin{equation}
    \text{ln} \left(U \cdot \frac{1}{\si{\volt}} \right) = a \cdot \text{ln}(k) + b
    %hier andere Beschriftung für a und b überlegen
\end{equation}

Dabei wird die Spannung $U$ durch ihre Einheit geteilt, da der Logarithmus nur für
für dimensionslose Größen definiert ist. Zudem ergibt sich umgeformt:

\begin{equation}
    U = \text{e}^b \cdot k^a \cdot \si{\volt}
    \label{eqn:Spannung}
\end{equation}

Da nun $U$ proportional zu den Fourierkoeffizienten ist, lässt sich die
Stärke deren Abfalls ermitteln.
Die Messdaten der Sägezahnspannung sind dazu in Tabelle \ref{tab:Messdaten1} aufgeführt und in 
Abbildung \ref{fig:Säge} gegeneinander aufgetragen.
Die Lineare Regression, die mittels python durchgeführt wurde, ergibt die Tangentenparameter:

\begin{align*}
    a_\text{S} &= \SI{-1.0052 \pm 0.0286}{} \\
    b_\text{S} &= \SI{07523 \pm 0.0451}{}
\end{align*}

Nach Formel \eqref{eqn:Spannung} ist zu erkennen, dass die Fourierkoeffizienten der Sägezahnspannung in etwa
proportional zu $\frac{1}{k}$ abfallen.

\begin{table}
    \centering
    \caption{Messdaten und deren Logarithmen der Sägezahnspannung}
    \label{tab:Messdaten1}
    \sisetup{table-format=2.1}
    \begin{tabular}{c c c c}
    \toprule
    $k$ & $\text{ln} (k)$ & $U \,/\, \si{\volt}$ & $\text{ln}(U \,/\, \si{\volt})$ \\
    \midrule
    1 & 0,000 & 2,040 &  0,713 \\
    2 & 0,693 & 1,100 &  0,095 \\
    3 & 1,099 & 0,740 & -0,301 \\
    4 & 1,386 & 0,488 & -0,717 \\
    5 & 1,609 & 0,440 & -0,821 \\
    6 & 1,792 & 0,352 & -1,044 \\
    7 & 1,946 & 0,296 & -1,217 \\
    8 & 2,079 & 0,280 & -1,273 \\
    9 & 2,197 & 0,216 & -1,532 \\
    \bottomrule
    \end{tabular}
\end{table} 

\begin{figure}
    \centering
    \includegraphics[scale=1.0]{content/plot1.pdf}
    \caption{Messwerte und lineare Regression für die Messwerte der Sägezahnspannung}
    \label{fig:Säge}
\end{figure}

Für die Rechtecksspannung sind die Messdaten in Tabelle \ref{tab:Messdaten2} aufgelistet und in Abbildung \ref{fig:Recht}
gegeneinander aufgetragen. Es ergeben sich die Regressionsparameter:

\begin{align*}
    a_\text{R} &= \SI{-1.0021 \pm 0.0209}{} \\
    b_\text{R} &= \SI{1.4205 \pm 0.0462}{}
\end{align*}

Nach Formel \eqref{eqn:Spannung} ist zu erkennen, dass die Fourierkoeffizienten der Rechtecksspannung in etwa
proportional zu $\frac{1}{k}$ abfallen.

\begin{table}
    \centering
    \caption{Messdaten und deren Logarithmen der Rechtecksspannung}
    \label{tab:Messdaten2}
    \sisetup{table-format=2.1}
    \begin{tabular}{c c c c}
    \toprule
    $k$ & $\text{ln} (k)$ & $U \,/\, \si{\volt}$ & $\text{ln}(U \,/\, \si{\volt})$ \\
    \midrule
    1 & 0,000 & 4,000 &  1,389 \\
    3 & 1,099 & 1,440 &  0,365 \\
    5 & 1,609 & 0,860 & -0,151 \\
    7 & 1,946 & 0,580 & -0,545 \\
    9 & 2,197 & 0,420 & -0,868 \\
   11 & 2,398 & 0,384 & -0,957 \\
   13 & 2,565 & 0,344 & -1,067 \\
   15 & 2,708 & 0,280 & -1,273 \\
   17 & 2,833 & 0,224 & -1,496 \\
   19 & 2,944 & 0,216 & -1,532 \\
    \bottomrule
    \end{tabular}
\end{table} 

\begin{figure}
    \centering
    \includegraphics[scale=1.0]{content/plot2.pdf}
    \caption{Messwerte und lineare Regression für die Messwerte der Rechtecksspannung}
    \label{fig:Recht}
\end{figure}

Für die Dreiecksspannung sind die Messdaten in Tabelle \ref{tab:Messdaten3} aufgelistet und in Abbildung \ref{fig:Drei}
gegeneinander aufgetragen. Es ergeben sich die Regressionsparameter:

\begin{align*}
    a_\text{D} &= \SI{-1.9681 \pm 0.0121}{} \\
    b_\text{D} &= \SI{0.9585 \pm 0.0227}{}
\end{align*}

Nach Formel \eqref{eqn:Spannung} ist zu erkennen, dass die Fourierkoeffizienten der Dreiecksspannung in etwa
proportional zu $\frac{1}{k^2}$ abfallen.

\begin{table}
    \centering
    \caption{Messdaten und deren Logarithmen der Dreiecksspannung}
    \label{tab:Messdaten3}
    \sisetup{table-format=2.1}
    \begin{tabular}{c c c c}
    \toprule
    $k$ & $\text{ln} (k)$ & $U \,/\, \si{\volt}$ & $\text{ln}(U \,/\, \si{\volt})$ \\
    \midrule
     1 & 0,000 & 2,560 &  0,940 \\
     3 & 1,099 & 0,312 & -1,165 \\
     5 & 1,609 & 0,112 & -2,189 \\
     7 & 1,946 & 0,056 & -2,882 \\
     9 & 2,197 & 0,036 & -3,324 \\
    11 & 2,398 & 0,023 & -3,772 \\
    13 & 2,565 & 0,017 & -4,075 \\
    \bottomrule
    \end{tabular}
\end{table} 

\begin{figure}
    \centering
    \includegraphics[scale=1.0]{content/plot3.pdf}
    \caption{Messwerte und lineare Regression für die Messwerte der Dreiecksspannung}
    \label{fig:Drei}
\end{figure}

\subsection{Fourier-Synthese}

Im zweiten Versuchsteil wird gezeigt, dass sich die verschiedenen Spannungsfunktionen
durch Superpositionen verschiedener Oberwellen, bzw Fourierkoeffizienten synthetisieren
lassen. Als Grundlage dessen dienen theoretische Vorberechnungen der jeweiligen
Fourier-Reihen, nach Formel \eqref{eqn:Entwicklung}, und der in diese eingesetzten
Fourierkoeffizienten $a_k$ und $b_k$.

Für die ungerade Sägezahnspannung sind die Fourierkoeffizienten:

\begin{align*}
    a_\text{k} &= 0 \\
    b_\text{k} &= - \frac{1}{2k} + \frac{1}{k}
\end{align*}

Diese ergeben in Formel \eqref{eqn:Entwicklung} eingesetzt die Fourier-Reihe
der Sägezahnspannung, die in Abbbildung \ref{fig:Theo1}  dargestellt ist.

\begin{figure}
    \includegraphics[scale = 1.0]{content/plot4.pdf}
    \caption{Theoretische Sägezahnfunktion}
    \label{fig:Theo1}
\end{figure}

Experimentell werden die Fourierkoeffizienten bzw. Oberwellenamplituden möglichst mit dem
Proportionalitätsfaktor $\frac{1}{k}$ abfallend eingestellt, wie in Tabelle \ref{tab:Messdaten4}
dargestellt. Dabei bezeichnen $U_\text{theo}$ die theoretisch einzustellenden und $U$ die
tatsächlich eingestellten Spannungamplituden. $\symup{\Delta}U$ bezeichnet die Abweichung der
beiden Werte voneinander. Es ergibt sich die Sägezahnspannung, welche in Abbildung \ref{fig:Ex1}
dargestellt ist.

\begin{table}
    \centering
    \caption{Einstellungen der Oberwellenamplituden für eine Sägezahnspannung}
    \label{tab:Messdaten4}
    \sisetup{table-format=2.1}
    \begin{tabular}{c c c c}
    \toprule
    $k$ & $U \,/\, \si{\volt}$ & $U_\text{theo} \,/\, \si{\volt}$ & $\symup{\Delta}U \,/\, \si{\percent})$ \\
    \midrule
    1 & 0,633 & 0,633 & 0,00 \\
    2 & 0,316 & 0,317 & 0,32 \\
    3 & 0,211 & 0,211 & 0,00 \\
    4 & 0,157 & 0,158 & 0,63 \\
    5 & 0,126 & 0,127 & 0,79 \\
    6 & 0,106 & 0,106 & 0,00 \\
    7 & 0,089 & 0,090 & 1,11 \\
    8 & 0,078 & 0,079 & 1,27 \\
    9 & 0,070 & 0,070 & 0,00 \\
   10 & 0,063 & 0,063 & 0,00 \\
    \bottomrule
    \end{tabular}
\end{table}

\begin{figure}
    \includegraphics[scale = 0.7]{content/MAP001.pdf}
    \caption{Experimentell synthetisierte Sägezahnspannung}
    \label{fig:Ex1}
\end{figure}



Für die ungerade Rechtecksspannung lauten die Fourierkoeffizienten:

\begin{align*}
    a_\text{k} &= 0 \\
    b_\text{k} &= 
        \begin{cases} 
            0, \text{ für k gerade} \\ \frac{4}{\pi k}, \text{ für k ungerade}
        \end{cases}
\end{align*}

Diese ergeben in Formel \eqref{eqn:Entwicklung} eingesetzt die Fourier-Reihe
der Rechtecksspannung, die in Abbbildung \ref{fig:Theo2}  dargestellt ist.

\begin{figure}
    \includegraphics[scale = 1.0]{content/plot5.pdf}
    \caption{Theoretische Rechtecksfunktion}
    \label{fig:Theo2}
\end{figure}

Die Fourierkoeffizienten der Rechtecksspannung werden analog zu denen der Sägezahnspannung
eingestellt , wie in Tabelle \ref{tab:Messdaten5} dargestellt. 
Es ergibt sich die Rechtecksspannung, welche in Abbildung \ref{fig:Ex2}
dargestellt ist.

\begin{table}
    \centering
    \caption{Einstellungen der Oberwellenamplituden für eine Rechtecksspannung}
    \label{tab:Messdaten5}
    \sisetup{table-format=2.1}
    \begin{tabular}{c c c c}
    \toprule
    $k$ & $U \,/\, \si{\volt}$ & $U_\text{theo} \,/\, \si{\volt}$ & $\symup{\Delta}U \,/\, \si{\percent})$ \\
    \midrule
     1 & 0,633 & 0,633 & 0.00 \\
     2 & 0,000 & 0,000 & 0.00 \\
     3 & 0,211 & 0,211 & 0.00 \\
     4 & 0,000 & 0,000 & 0.00 \\
     5 & 0,126 & 0,127 & 0.79 \\
     6 & 0,000 & 0,000 & 0.00 \\
     7 & 0,089 & 0,090 & 1.11 \\
     8 & 0,000 & 0,000 & 0.00 \\
     9 & 0,070 & 0,070 & 0.00 \\
    10 & 0,000 & 0,000 & 0.00 \\
    \bottomrule
    \end{tabular}
\end{table}

\begin{figure}
    \centering
    \includegraphics[scale=0.7]{content/MAP002.pdf}
    \caption{Experimentell synthetisierte Rechtecksspannung}
    \label{fig:Ex2}
\end{figure}



Für die gerade Dreiecksspannung lauten die Fourierkoeffizienten:

\begin{align*}
    a_\text{k} &= 0 
        \begin{cases} 
            0, \text{ für k gerade} \\ \frac{8}{\pi^2 k^2}, \text{ für k ungerade}
        \end{cases} \\
    b_\text{k} &= 0
\end{align*}

Diese ergeben in Formel \eqref{eqn:Entwicklung} eingesetzt die Fourier-Reihe
der Dreiecksspannung, die in Abbbildung \ref{fig:Theo3}  dargestellt ist.

\begin{figure}
    \includegraphics[scale = 1.0]{content/plot6.pdf}
    \caption{Theoretische Rechtecksfunktion}
    \label{fig:Theo3}
\end{figure}

Die Fourierkoeffizienten der Dreiecksspannungen werden im Gegensatz zu denen
der vorherigen mit einem Abfall proportional zu $\frac{1}{k^2}$ eingestellt,
wie in Tabelle \ref{tab:Messdaten6} dargestellt. Die abgebildeten
Größen ergeben sich analog zu den Tabellen der anderen beiden Spannungen.
Es ergibt sich schließlich die Dreiecksspannung, welche in Abbildung \ref{fig:Ex3}
dargestellt ist.

\begin{table}
    \centering
    \caption{Einstellungen der Oberwellenamplituden für eine Dreiecksspannung}
    \label{tab:Messdaten6}
    \sisetup{table-format=2.1}
    \begin{tabular}{c c c c}
    \toprule
    $k$ & $U \,/\, \si{\volt}$ & $U_\text{theo} \,/\, \si{\volt}$ & $\symup{\Delta}U \,/\, \si{\percent})$ \\
    \midrule
    1 & 0,617 & 0,617 & 0.00 \\
    2 & 0,000 & 0,000 & 0.00 \\
    3 & 0,068 & 0,069 & 1.45 \\
    4 & 0,000 & 0,000 & 0.00 \\
    5 & 0,025 & 0,025 & 0.00 \\
    6 & 0,000 & 0,000 & 0.00 \\
    7 & 0,013 & 0,013 & 0.00 \\
    8 & 0,000 & 0,000 & 0.00 \\
    9 & 0,008 & 0,008 & 0.00 \\
    \bottomrule
    \end{tabular}
\end{table}

\begin{figure}
    \centering
    \includegraphics[scale=0.7]{content/MAP003.pdf}
    \caption{Experimentell synthetisierte Dreiecksspannung}
    \label{fig:Ex3}
\end{figure}




