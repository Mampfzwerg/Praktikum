\section{Diskussion}
\label{sec:Diskussion}

Im ersten Teil der Auswertung wurde der Abfall der Koeffizienten untersucht. 
Dabei wurde festgestellt, dass diese bei Sägezahn- und Rechtecksignal fast 
exakt mit $\frac{1}{k}$ und bei dem Dreiecksignal ziemlich genau mit $\frac{1}{k²}$
abfallen. Genau das ist auch anhand der berechneten Koeffizienten der Signale 
zu erwarten. 
\\
Im zweiten Teil der Auswertung wird die Fourier-Synthese untersucht. Es stellt 
sich heraus, dass die Synthese des Dreiecksignals gut funktioniert und 
keine gravierenden Abweichung zwischen dem berechneten und tatsächlichen Kurvenverlauf
festgestellt werden kann. Bei der Synthese des Rechteck- und Sägezahnsignals 
hingegen treten kleine Abweichungen zu den berechneten Funktionen auf. Diese 
können durch geringfügige Phasendifferenzen erklärt werden, welche dadurch zustande
kommen, dass die Lissajous-Figuren bei größerem $k$ immer schwerer erkennbar und 
einstellbar sind oder aber aufgrund des Gibbschen Phänomens. 
\\
Insgesamt lässt sich also sagen, dass sowohl Fourier-Analyse als auch Fourier-Synthese
im Experiment den theoretischen Vorhersagen entsprechen und somit das 
Fouriersche Theorem verifizieren. 