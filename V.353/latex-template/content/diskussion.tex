\section{Diskussion}
\label{sec:Diskussion}

Durch Messung der Auslenkung bzw. Spannung $U_C$ bei der Auf-/Entladung des Kondensators 
ergibt sich die Zeitkonstante als

\begin{equation*}
RC_1 = -\frac{1}{a} = \SI{1.09+-0.05}{\milli\second} \; .
\end{equation*}

Beim zweiten Versuchsteil wird die Zeitkonstante durch Messung der Frequenzabhängigkeit der
Amplitude bestimmt und ergibt sich als

\begin{equation*}
RC_2 = \SI{8.64+-0.34}{\milli\second} \; .
\end{equation*}

Zuletzt ergibt sich für die Messung der Frequenzabhängigkeit der Phasenverschiebung:

\begin{equation*}
    RC_3 = \SI{5.8+-0.9}{\milli\second} \; .
\end{equation*}

Dabei sind die statistischen Fehler aller Größen verhältnismäßig gering und können durch das ungenaue
Ablesen und schwankende Werte erklärt werden.

Die Messwerte der ersten
beiden Messungen liegen gut auf ihren Ausgleichgraphen - für $RC_1$ logarithmisch aufgetragen auf einer Geraden,
für $RC_2$ auf einer Funktion ähnlich zu \eqref{eqn:Amplitude}. Bei der letzten Messung ist die statistische
Abweichung von $RC_3$ mit $0.9$ fast dreimal so groß, wie die der ersten beiden Messungen. Das könnte an der doppelt
so fehleranfälligen Differenzmessung der Phasenverschiebung liegen. Auch passen die Messwerte schlechter zum Ausgleichsgraphen
gemäß \eqref{eqn:Phase}.\\
Anhand der Abbildungen \ref{fig:Sinus}, \ref{fig:Dreieck} und \ref{fig:Rechteck} ist zu
erkennen, dass der RC-Kreis als Integrator genutzt werden kann. Dies ist daran zu erkennen, 
dass die Extrema der integrierten Funktionen stets in den Zeiten der Nullpunkte der 
Ursprungssignale liegen. Außerdem ist an dem Beispiel der Sinusspannung gut zu erkennen, 
dass sich als integrierte Funktion eine Cosinusspannung ergibt.