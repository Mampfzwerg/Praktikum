\section{Diskussion}
\label{sec:Diskussion}

Zur besseren Übersicht der Ergebnisse sind die experimentell und theoretisch bestimmten Werte der Absorptionskoeffizienten
mit der jeweiligen Abweichung in Tabelle \ref{tab:dtab1} dargestellt.
\FloatBarrier
\begin{table}[h]
    \centering
    \caption{Experimentell und theoretisch bestimmte Werte der Absorptionskoeffizienten von Blei und Eisen.}
    \label{tab:atab2}
    \begin{tabular}{l S[table-format=3.3] @{${}\pm{}$} S[table-format=1.4] S[table-format=2.3] S[table-format=2.2]}
        \toprule
        {} & \multicolumn{2}{c}{$\mu_\text{Experiment} / \, \si[per-mode=reciprocal]{\per\meter}$} & {$\mu_\text{Compton} / \, \si[per-mode=reciprocal]{\per\meter}$} & {Abweichung / \%} \\
        \midrule
        {Blei}  &  -107,635 & 2,317 & 69,351 & 55,20 \\
        {Eisen} &  -45,594  & 0,873 & 56,640 & 21,27 \\
        \bottomrule
    \end{tabular}
\end{table}
\FloatBarrier
\noindent
Die Werte des experimentell bestimmten Absorptionskoeffizienten von Blei und Eisen weichen deutlich von den berechneten
Werten der Absorptionskoeffizienten $\mu_\text{Compton}$ ab. Die dieser in beiden Fällen unterhalb des experimentell 
bestimmten Wertes liegt, ist zu schlussfolgern, dass bei dieser Messung der Photoeffekt der dominierende Absorptionsmechanismus ist. 
Die Werte von $N(0) = \num{4,912(72)} \, \si[per-mode=reciprocal]{\per\second}$ für Blei und $N(0) = \num{4,933(27)} \, \si[per-mode=reciprocal]{\per\second}$
für Eisen unterscheiden sich mit einer Abweichung von 0,43\% nur geringfügig, was auf eine hohe Genauigkeit der Messung schließen lässt.
Messfehler können jedoch durch das verwendete Absorbermaterial entstanden sein, welches möglicherweise nicht überall
die gleiche Dicke aufweist.

Bei der $beta$-Strahlung tritt der Nulleffekt besonders bei niedrigen Zählraten aus, 
deswegen ist dieser genau auszumessen. Auch hier kann durch längeres Messen der Fehler verringert werden. 
Es ist anzumerken, dass die Schichtdicke soweit angepasst werden soll, dass die Zählrate annähernd 
gleich bleibt, bis auf statistische Schwankungen. 