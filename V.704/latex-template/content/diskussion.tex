\section{Diskussion}
\label{sec:Diskussion}

Bei der $beta$-Strahlung tritt der Nulleffekt besonders bei niedrigen Zählraten aus, 
deswegen ist dieser genau auszumessen. Auch hier kann durch längeres Messen der Fehler verringert werden. 
Es ist anzumerken, dass die Schichtdicke soweit angepasst werden soll, dass die Zählrate annähernd 
gleich bleibt, bis auf statistische Schwankungen. 