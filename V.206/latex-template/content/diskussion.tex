\section{Diskussion}
\label{sec:Diskussion}

Es wurden die Güte $\nu$ und der Massendurchsatz 
$\frac{\symup{d}m}{\symup{d}t}$ der Wärmepumpe, sowie die 
mechanische Leistung $N_\text{mech}$ des Kompressors bestimmt. 

Es ist zu bemerken, dass die realen Güteziffern stark 
von der idealen abweichen. Dafür gibt es mehrere Gründe. 
Zunächst kann die Isolierung den Wärmeaustausch mit der 
Umgebung nicht vollständig verhindern. Außerdem waren
die Skalen auf den Manometern recht ungenau abzulesen. 
Darüber hinaus erfolgt die Kompression nicht absolut 
adiabatisch, sodass Energieverluste auftreten. 

Desweiteren weicht der errechnete Wert der Verdampfungswärme
$L = \SI{142}{\joule\per\gram}$ um \si{15.08}\% von dem Literaturwert 
$L_\text{lit}=\SI{167.22}{\joule\per\gram}$ [2] ab. Das ist 
verglichen mit den Abweichungen der Güteziffern ein recht gutes 
Ergebnis.