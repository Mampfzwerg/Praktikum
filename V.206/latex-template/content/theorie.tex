\section{Theorie}
\label{sec:Theorie}

Ziel des Versuches ist es, die Kenngröße der Wärmepumpe zu 
ermitteln, indem der Transport von Wärmeenergie zwischen
zwei Wärmereservoiren untersucht wird.

\subsection{Das Prinzip der Wärmepumpe}

Ohne äußere Einflüsse findet Temperaturänderung in Form von Wärmeabgabe vom heißeren zum kälteren
Körper oder Medium statt. Durch aufgewandte (mechanische) Arbeit $A$ kann dieser Prozess jedoch auch in anderer
Richtung ablaufen. Gemäß dem ersten Hauptsatz der Thermodynamik beträgt die an das wärmere Medium 
abgegebene Wärmemenge $Q_1$, die von dem kühleren Medium aufgenommene Wärmemeng $Q_2$, zuzüglich der Arbeit $A$.
Dementsprechend gilt

\begin{equation}
    Q_1 = Q_2 + A
    \label{eqn:Wärm}
\end{equation}

Die Güteziffer $\nu$ der Wärmepumpe gibt dabei das Verhältnis der abgegebenen Wärmemenge $Q_1$ zur aufgewandten
Arbeit $A$ an

\begin{equation}
    \nu = \frac{Q_1}{A}
    \label{eqn:Güte}
\end{equation}

Der zweite Hauptsatz der Thermodynamik führt für die reduzierten Wärmemengen zu der Beziehung, dass deren Summe 
$\int \frac{\symup{d}Q}{T}$ null beträgt. Aus dieser folgt

\begin{equation}
    \frac{Q_1}{T_1} - \frac{Q_2}{T_2} = 0
    \label{eqn:redWärm}
\end{equation}

Allerdings muss es sich für diese Beziehung um einen idealen reversiblen, d.h. umkehrbaren Prozess handeln. 
Vom Ideal abweichend, gilt für die technische Anwendung also die Ungleichung

\begin{equation}
    \frac{Q_1}{T_1} - \frac{Q_2}{T_2} > 0
    \label{eqn:ungWärme}
\end{equation}

Aus \ref{eqn:Wärm}  und \ref{eqn:redWärm} folgt

\begin{equation*}
    Q_1 = A + \frac{T_2}{T_1} \cdot Q_1 \; \text{,}
\end{equation*}

aus \ref{eqn:Güte}, für einen reversiblen Vorgang, die ideale Güte

\begin{equation}
    \nu_\text{id} = \frac{Q_1}{A} = \frac{T_1}{T_1 - T_2}
    \label{eqn:ideal}
\end{equation}

und aus \ref{eqn:Wärm} und \ref{eqn:ungWärme} für die Güte der realen Wärmepumpe

\begin{equation}
    \nu_\text{real} < \frac{T_1}{T_1 - T_2} \; \text{.}
    \label{eqn:real}
\end{equation}

Aus \ref{eqn:ideal} und \ref{eqn:real} ist abzulesen, dass die Wärmepumpe für kleine
Temperaturdifferenzen $T_1 - T_2$ am effizientesten arbeitet, die aufgewandte Arbeit $A$ zur Wärmeübertragung
also minimal ist.

\subsection{Die Arbeitsweise der Wärmepumpe}

In der Wärmepumpe fungiert ein reales Gas als Transportmedium, welches bei Wärmeaufnahme verdampft und die Wärme
durch Kondensation wieder abgibt, die Wärmeenergie infolgedessen als Phasenumwandlungsenergie transportiert.
Vorteilhaft ist daher die Verwendung von Gasen hoher Kondensationswärme. 

Nach dem Aufbau der Wärmepumpe 
(Abbildung  ) sorgt der Kompressor K für einen Kreislauf, den das Transportgas und somit sowohl beide
Wärmereservoires, als auch ein Drosselventil D durchläuft. An diesem entsteht ein Druckunterschied $p_\text{b} 
- p_\text{a}$. Dabei ist das Transportgas bei Druck $p_\text{b}$ und Temperatur $T_1$ flüssig und bei $p_\text{a}$
und $T_2$ gasförmig.

Dem kälteren Reservoire 2 wird durch das Verdampfen des Transportgases die Verdampfungswärme $L$ pro gramm 
entzogen. Darauf wird das Gas im Kompressor K adiabatisch komprimiert, sodass dessen Druck und Temperatur steigen
und es schließlich die Kondensationswärme $L$ pro gramm an das Reservoire 1 abgibt.

Weitere nötige Komponenten der Wärmepumpe sind ein Reiniger R, der die Blasen im flüssigen Medium entfernt, sowie
ein Steuerungselement S, welches mit dem Drosselventil D gekoppelt ist und das schädliche Eindringen flüssigen 
Mediums in den Kompressor K über Kontrolle der Temperaturdifferenz am Ein- und Ausgang des Reservoires 2 verhindert.









